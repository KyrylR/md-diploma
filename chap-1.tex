\chapter{ТЕОРЕТИЧНЕ ПІДҐРУНТЯ КРИПТОСИСТЕМИ}\label{ch:-1:---}

У цьому розділі викладено основні теоретичні положення, що лежать в основі розробки та аналізу симетричної криптосистеми, запропонованої у цій роботі.


\section{Вступ до основ криптографії}
\label{sec:crypto_intro}

Криптографія пройшла шлях від простих шифрів заміни, що використовувалися ще в античності (наприклад, шифр Цезаря),
до сучасних протоколів захисту інформації (зокрема, TLS, системи наскрізного шифрування).
Основними завданнями криптографії є забезпечення конфіденційності (захист від несанкціонованого доступу),
цілісності (контроль змін) та автентифікації (верифікація ідентичності).

Побудова надійних криптосистем ґрунтується на двох фундаментальних принципах:
\begin{itemize}
    \item \emph{Принцип Керкгоффса}~\cite{Kerckhoffs83}: Безпека криптосистеми не повинна залежати від секретності алгоритму;
    достатньо зберігати в таємниці лише ключ.
    \item \emph{Досконала секретність за Шенноном}~\cite{Shannon49}: Шифр вважається досконалим, якщо шифротекст не містить жодної
    інформації про відкритий текст (наприклад, одноразовий блокнот за умови випадковості ключа).
\end{itemize}

К. Шеннон також увів поняття \emph{перемішування} (confusion) та \emph{розсіювання} (diffusion), які реалізуються у сучасних
блокових шифрах (наприклад, AES) через мережі підстановок і перестановок~\cite{Shannon49}.

Ці високорівневі концепції лежать в основі розробленої криптосистеми.

\subsection{Базові поняття та означення}
\label{subsec:crypto_basics}

Для повного розуміння розробленої системо розглянемо наступні означення~\cite{Mao04,BerczesEtAl14}:

\begin{description}
    \item[Криптосистема:] Трійка алгоритмів \((\mathsf{KeyGen},\mathsf{Encrypt},\mathsf{Decrypt})\), що працюють за поліноміальний час відносно параметра безпеки \(n\).
    \(\mathsf{KeyGen}\) генерує ключі; \(\mathsf{Encrypt}(K,P)\) перетворює відкритий текст \(P\) у шифротекст \(C\); \(\mathsf{Decrypt}(K,C)\) відновлює \(P\).
    \item[Відкритий текст (\(P\)):] Початкове повідомлення.
    \item[Шифротекст (\(C\)):] Закодований результат, що приховує \(P\).
    \item[Ключ (\(K\)):] Секретний параметр, що визначає відображення шифрування та розшифрування.
    \item[Шифрування:] Операція \(C \leftarrow \mathsf{Encrypt}(K,P)\).
    \item[Розшифрування:] Операція \(P \leftarrow \mathsf{Decrypt}(K,C)\), причому \\ \(\mathsf{Decrypt}(K, \mathsf{Encrypt}(K,P)) = P\).
    \item[Криптоаналіз:] Методи відновлення \(K\) або \(P\) з \(C\) без авторизованого доступу.
\end{description}

Для кращого розуміння вище наведених визначень розглянемо наступний приклад.

\begin{example}[Шифр Вернама / Одноразовий блокнот]
    \label{ex:otp_vernam}
    Цей симетричний шифр, також відомий як одноразовий блокнот (One-Time Pad, OTP), був запропонований Гільбертом Вернамом.
    Нехай відкритий текст \(P\), шифротекст \(C\) та ключ \(K\) є бінарними послідовностями однакової довжини \(n\), тобто \(P, C, K \in \{0, 1\}^n\).
    Операції шифрування та розшифрування виконуються побітово за допомогою операції XOR (\(\oplus\), що еквівалентно додаванню за модулем 2):
    \begin{itemize}
        \item Шифрування: \(C = P \oplus K\) (тобто \(C_i = P_i \oplus K_i\) для \(i=1, \dots, n\)).
        \item Розшифрування: \(P = C \oplus K\) (оскільки \(C \oplus K = (P \oplus K) \oplus K = P \oplus (K \oplus K) = P \oplus 0 = P\)).
    \end{itemize}
    Шифр Вернама забезпечує досконалу секретність за Шенноном, тобто є теоретично невразливим, за умови суворого дотримання таких вимог до ключа \(K\):
    \begin{enumerate}
        \item Ключ \(K\) має бути абсолютно випадковою послідовністю.
        \item Довжина ключа \(|K|\) повинна бути не меншою за довжину відкритого тексту \(|P|\).
        \item Ключ \(K\) повинен використовуватися для шифрування лише одного повідомлення (одноразове використання).
    \end{enumerate}
    Порушення будь-якої з цих умов робить шифр вразливим.
    Основною практичною складністю є генерація та безпечна передача довгих, випадкових одноразових ключів~\cite{Shannon49}.
\end{example}

Криптографічні системи можемо поділити на \emph{симетричні криптосистеми}, де один і той самий секретний ключ \(K\) використовується для шифрування і розшифрування.
До прикладів можемо віднести: шифр Вернама (який наведено як приклад вище), AES, ChaCha20.
Та на \emph{Асиметричні криптосистеми} (з відкритим ключем), де для шифрування використовується відкритий ключ \(K_{\text{pub}}\), а для розшифрування — приватний ключ \(K_{\text{priv}}\).
До прикладів можемо віднести RSA, ElGamal encryption.

Запропонована система належить до симетричних і ґрунтується на сюр'єктивних відображеннях скінченних кілець та їх ізоморфізмів з використанням систем лінійних рівнянь над кільцями лишків.

\subsection{Мотивація та підхід створення криптосистеми}
\label{subsec:motivation}

Метою цієї роботи є розробка криптосистеми, що базується на об'єктах порівняно невеликих розмірів і забезпечує достатній рівень стійкості до злому.
Можна виділити наступні ключові характеристики системи:

\begin{enumerate}
    \item \emph{Обчислювальна ефективність:} Використання малого модуля \(k\) зменшує вартість арифметичних операцій.
    \item \emph{Простота реалізації:} Скінченні кільця \(\mathbb{Z}_k\) легко реалізуються на пристроях з обмеженими ресурсами (наприклад, смарт-карти, IoT).
    \item \emph{Комбінаторна стійкість:} Кількість сюр'єктивних гомоморфізмів \(\mathbb{Z}_k \to \mathbb{Z}_\ell\) та ізоморфізмів кілець швидко зростає зі зростанням кількості дільників \(k\), що ускладнює криптоаналіз.
    \item \emph{Стійкість до статистичних атак:} Вбудовування відкритого тексту у системи лінійних рівнянь над \(\mathbb{Z}_k\) ускладнює застосування частотного аналізу та подібних методів.
\end{enumerate}

Запропонована схема розвиває ідеї симетричного протоколу обміну~\cite{KryvyiEtAl22}, доповнюючи їх кільцевими перетвореннями та афінними відображеннями систем лінійних рівнянь.


\section{Основи теорії кілець}
\label{sec:ring_theory}

Кільця становлять алгебраїчну основу для побудови криптосистеми.
У цьому підрозділі наведено їхні означення, властивості та приклади, що мають значення для подальшого викладу.

\subsection{Означення та властивості кілець}
\label{subsec:ring_definition}

\begin{definition}
    \label{def:ring}
    \emph{Кільце} \((R,+,\cdot)\) — це множина \(R\) з двома бінарними операціями, для яких виконуються такі умови:
    \begin{enumerate}
        \item \((R,+)\) — абелева група з нульовим елементом \(0\) та протилежним елементом \(-a\) для кожного \(a\in R\).
        \item \((R,\cdot)\) — моноїд з одиницею \(1\): множення асоціативне (\(a(bc)=(ab)c\)), існує \(1\in R\) таке, що \(1\cdot a = a\cdot 1 = a\) для всіх \(a\in R\).
        \item \emph{Розподільність:} для всіх \(a,b,c\in R\)
        \[
            a\cdot(b + c) = ab + ac, \quad (b + c)\cdot a = ba + ca.
        \]
    \end{enumerate}
    Якщо множення комутативне (\(ab=ba\)), кільце називають \emph{комутативним}.
    Якщо \(R\) скінченне, комутативне та має одиницю \(1\neq0\), таке кільце називають \emph{скінченним комутативним кільцем з одиницею}.
\end{definition}

Основні поняття, пов'язані з кільцями:
\begin{description}
    \item[Дільник нуля:] Елемент \(a\neq0\), для якого існує \(b\neq0\) такий, що \(ab=0\).
    \item[Дільник одиниці (оборотний елемент):] Елемент \(u\in R\), для якого існує обернений \(u^{-1}\) такий, що \(uu^{-1}=u^{-1}u=1\).
    Множина всіх дільників одиниці утворює групу відносно множення, позначається \(R^\times\).
    \item[Характеристика:] Найменше натуральне \(n>0\), для якого \(n\cdot1 = 0\) (тобто \(1+\dots+1=0\), \(n\) разів).
    Якщо такого \(n\) не існує, характеристика дорівнює нулю.
\end{description}

\subsection{Кільця лишків за модулем \(k\) (\(\mathbb{Z}_k\))}
\label{subsec:residue_rings}

Нехай \(k\ge2\) — ціле число.
Кільце лишків за модулем \(k\) — це множина класів конгруентності за модулем \(k\):
\[
    \mathbb{Z}_k = \{\bar{0},\bar{1},\dots,\overline{k-1}\}, \quad \text{де } \bar{a} = a + k\mathbb{Z} = \{x\in\mathbb{Z} : x \equiv a \pmod{k}\}.
\]
Операції визначаються так:
\[
    \bar{a} + \bar{b} = \overline{a + b}, \quad \bar{a}\cdot\bar{b} = \overline{a\cdot b},
\]
тобто додавання та множення виконуються за модулем \(k\).
\((\mathbb{Z}_k, +, \cdot)\) — скінченне комутативне кільце з одиницею \(\bar{1}\).

\begin{example}[Операції в \(\mathbb{Z}_8\)]
    \label{ex:z8_ops}
    У кільці \(\mathbb{Z}_8\):
    \begin{gather*}
        \bar{3} + \bar{6} = \overline{3 + 6} = \overline{9} \equiv \bar{1} \pmod{8}.\\
        \bar{3}\cdot\bar{6} = \overline{3 \cdot 6} = \overline{18} \equiv \bar{2} \pmod{8}.\\
    \end{gather*}
    Дільники одиниці: \(\mathbb{Z}_8^\times = \{\bar{a} \in \mathbb{Z}_8 : \gcd(a,8)=1\} = \{\bar{1},\bar{3},\bar{5},\bar{7}\}\).
    Дільники нуля: \(\{\bar{a} \in \mathbb{Z}_8 \setminus \{\bar{0}\} : \gcd(a,8)>1\} = \{\bar{2},\bar{4},\bar{6}\}\).
    Наприклад, \(\bar{2}\cdot\bar{4} = \bar{8} \equiv \bar{0} \pmod{8}\).
\end{example}

\begin{example}[Ізоморфізм кілець]
    \label{ex:crt_iso}
    За китайською теоремою про лишки, якщо \(k={k_1k_2}\) і \(\gcd(k_1,k_2)=1\), то кільця \(\mathbb{Z}_k\) та \(\mathbb{Z}_{k_1} \times \mathbb{Z}_{k_2}\) є ізоморфними: \(\mathbb{Z}_k \cong \mathbb{Z}_{k_1} \times \mathbb{Z}_{k_2}\).
    Наприклад, \(\mathbb{Z}_6 \cong \mathbb{Z}_2 \times \mathbb{Z}_3\).
    Ізоморфізм \(\psi: \mathbb{Z}_6 \to \mathbb{Z}_2 \times \mathbb{Z}_3\) задається як \(\psi(\bar{a}) = (\overline{a \bmod 2}, \overline{a \bmod 3})\).
    Наприклад, \(\psi(\bar{5}) = (\overline{5 \bmod 2}, \overline{5 \bmod 3}) = (\bar{1}, \bar{2})\).
\end{example}

Позначимо через \(G_k\) будь-яке кільце, ізоморфне \(\mathbb{Z}_k\), з фіксованим ізоморфізмом \(\varphi: G_k \to \mathbb{Z}_k\).

\subsection{Мультиплікативна група дільників одиниці}
\label{subsec:ring_units_group}

Множина дільників одиниці \(\mathbb{Z}_k^\times\) утворює групу відносно множення за модулем \(k\):
\[
    \mathbb{Z}_k^\times = \{\bar{a}\in\mathbb{Z}_k : \gcd(a,k)=1\}.
\]
Ця група є абелевою.
Її порядок дорівнює \(\varphi(k)\), де \(\varphi\) — функція Ойлера.

\begin{example}[Група \(\mathbb{Z}_{10}^\times\)]
    \label{ex:z10_units}
    У \(\mathbb{Z}_{10}\), дільники одиниці: \(\mathbb{Z}_{10}^\times = \{\bar{1},\bar{3},\bar{7},\bar{9}\}\).
    Порядок групи \(\varphi(10) = 4\).
    Операція — множення за модулем 10.
    Наприклад, \(\bar{3}\cdot\bar{7} = \overline{21} \equiv \bar{1} \pmod{10}\), отже \(\bar{7} = \bar{3}^{-1}\).
    Група \(\mathbb{Z}_{10}^\times\) є циклічною, оскільки \(\bar{3}\) є генератором групи.
\end{example}

\begin{theorem}[Гаус~\cite{Shoup08}]
    \label{thm:cyclic_units}
    Група \(\mathbb{Z}_k^\times\) є циклічною тоді і тільки тоді, коли \(k=1,2,4,p^m\) або \(k=2p^m\), де \(p\) — непарне просте число, \(m\geq1\).
\end{theorem}

\subsection{Відображення між кільцями}
\label{subsec:ring_mappings}

Розглядаються такі типи відображень між кільцями:

\begin{description}
    \item[Гомоморфізм \(\phi: R\to S\):] Відображення, що зберігає операції: \(\phi(a+b)=\phi(a)+\phi(b)\) та \(\phi(a\cdot b)=\phi(a)\cdot\phi(b)\) для всіх \(a,b\in R\).
    \item[Ізоморфізм \(\varphi: R\to S\):] Бієктивний гомоморфізм.
    Якщо існує ізоморфізм, кільця \(R\) і \(S\) структурно еквівалентні (\(R\cong S\)).
    \item[Сюр'єкція (епіморфізм) \(\psi: R\to T\):] Гомоморфізм, образ якого збігається з усім \(T\) (\(\mathrm{Im}\,\psi = T\)).
    \item[Бієкція \(\psi_1: R\to R'\):] Взаємно однозначне відображення між множинами \(R\) та \(R'\), яке не обов'язково зберігає операції кілець.
\end{description}

\subsection{Ідеали та фактор-кільця}
\label{subsec:factor_rings}

Нехай \(R\) — кільце. Для побудови фактор-кільця необхідно ввести поняття ідеалу.

\textbf{Ідеал.} Підмножина \(I \subseteq R\) називається \emph{ідеалом} кільця \(R\), якщо виконуються такі умови:
\begin{itemize}
    \item \(I\) є підгрупою відносно додавання: для будь-яких \(a, b \in I\) маємо \(a-b \in I\);
    \item \(I\) замкнена відносно множення на довільний елемент кільця: для будь-яких \(r \in R\), \(a \in I\) маємо \(ra \in I\) і \(ar \in I\).
\end{itemize}
Ідеал можна розглядати як "узагальнення" поняття кратних у кільці цілих чисел: наприклад, множина всіх чисел, кратних \(n\), є ідеалом у \(\mathbb{Z}\).

\textbf{Фактор-кільце.} Нехай \(I\) — ідеал у кільці \(R\). \emph{Фактор-кільце} \(R/I\) — це множина всіх класів суміжності за ідеалом \(I\), тобто множина підмножин вигляду:
\[
    a + I = \{a + r : r \in I\}, \quad \text{де } a \in R.
\]
Кожен елемент \(R\) належить рівно одному такому класу.

Операції додавання та множення на \(R/I\) визначаються так:
\begin{gather*}
    (a + I) + (b + I) = (a + b) + I, \\
    (a + I) \cdot (b + I) = (ab) + I.
\end{gather*}
Ці операції коректно визначені, тобто не залежать від вибору представників класів.

\textbf{Властивість.} Фактор-кільце \(R/I\) саме є кільцем, а природне відображення \(\pi: R \to R/I\), \(\pi(a) = a + I\), є гомоморфізмом кілець з ядром \(I\).

\textbf{Теорема про ізоморфізм.} Якщо \(\psi: R \to T\) — сюр'єктивний гомоморфізм кілець, то фактор-кільце \(R/\ker\psi\) ізоморфне образу \(\psi\), тобто \(R/\ker\psi \cong \mathrm{Im}(\psi)\). Зокрема, якщо \(\psi\) сюр'єктивне, то \(R/\ker\psi \cong T\).

Таким чином, фактор-кільце дозволяє \texttt{"}розділити\texttt{"} кільце на класи за ідеалом і отримати нову алгебраїчну структуру, яка часто має простішу або більш зручну для застосування будову.

\section{Системи лінійних рівнянь над кільцями}
\label{sec:sle_theory}

У цьому підрозділі розглядаються системи лінійних рівнянь (СЛР) над скінченними комутативними кільцями.

\subsection{Означення та матричний запис}
\label{subsec:sle_definition}

Нехай \(R = \mathbb{Z}_m\) — скінченне комутативне кільце з одиницею.
\emph{Система з \(n\) лінійних рівнянь з \(n\) невідомими} над \(R\) має вигляд:
\[
    \begin{cases}
        a_{11}x_1 + a_{12}x_2 + \cdots + a_{1n}x_n \;\equiv\; b_1 \pmod{m},\\
        a_{21}x_1 + a_{22}x_2 + \cdots + a_{2n}x_n \;\equiv\; b_2 \pmod{m},\\
        \quad\vdots\\
        a_{n1}x_1 + a_{n2}x_2 + \cdots + a_{nn}x_n \;\equiv\; b_n \pmod{m},
    \end{cases}
\]
де коефіцієнти \(a_{ij} \in \mathbb{Z}_m\), вільні члени \(b_i \in \mathbb{Z}_m\), а невідомі \(x_j\) шукаються в \(\mathbb{Z}_m\).

У матричному записі ця система має вигляд:
\[
    A\,x \;\equiv\; b \pmod{m},
\]
де \(A = (a_{ij})\) — матриця розміру \(n\times n\) з елементами з \(\mathbb{Z}_m\), \(x = (x_j)\) — стовпчиковий вектор невідомих розміру \(n\times 1\), \(b = (b_i)\) — стовпчиковий вектор вільних членів розміру \(n\times 1\).

\subsection{Лінійні діофантові рівняння та конгруенції}
\label{subsec:diophantine}

\emph{Лінійне діофантове рівняння} — це рівняння вигляду \(a_1 x_1 + \dots + a_k x_k = c\) з цілими коефіцієнтами \(a_i, c \in \mathbb{Z}\), де шукаються цілі розв'язки \((x_1,\dots,x_k)\)~\cite{KameswariEtAl21}.
Система лінійних конгруенцій за модулем \(m\), як визначено вище, тісно пов'язана з діофантовими рівняннями.
Кожна конгруенція \( \sum_j a_{ij} x_j \equiv b_i \pmod{m} \) еквівалентна лінійному діофантовому рівнянню \( \sum_j a_{ij} x_j - m y_i = b_i \) для деякого цілого числа \(y_i\).

Окрема лінійна конгруенція \(a x \equiv c \pmod{m}\) має розв'язок для \(x \in \mathbb{Z}_m\) тоді й лише тоді, коли \(d = \gcd(a,m)\) ділить \(c\).
Якщо розв'язок існує, то їх рівно \(d\) за модулем \(m\)~\cite{Kryvyi21}.

Розв'язання систем лінійних рівнянь над \(\mathbb{Z}_m\) у цій роботі ґрунтується на алгоритмах, описаних у~\cite{Kryvyi21}.

\subsection{Перетворення систем}
\label{subsec:sle_transformations}

Для обфускації системи \(A\,x \equiv b \pmod m\) можна застосовувати афінне перетворення змінних, яке задається оберненою матрицею \(B \in \mathrm{GL}_n(\mathbb{Z}_m)\) та вектором зсуву \(a \in \mathbb{Z}_m^n\):
\[
    x = B y + a.
\]
Підставляючи це у вихідну систему \(Ax \equiv b \pmod m\), отримуємо:
\[
    (A B) y \equiv b - A a \pmod{m}
\]
Позначимо \(A' = A B\) та \(b' = b - A a\).
Отримана перетворена система має вигляд \(A' y \equiv b' \pmod m\).

Інший тип перетворення — домноження обох частин системи зліва на обернену матрицю \(C \in \mathrm{GL}_n(\mathbb{Z}_m)\):
\[
    (C A) x \equiv C b \pmod m.
\]

Комбінування цих перетворень та їх композиція з ізоморфізмами чи гомоморфізмами кілець можуть змінювати подання системи.
