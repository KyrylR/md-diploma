\chapter*{ВСТУП}
\phantomsection
\addcontentsline{toc}{chapter}{ВСТУП}

{\Large

Оцінка сучасного стану об’єкта розроблення. Правильно організована
навчально-пізнавальна діяльність --- це головне джерело розвитку пізнавальних
інтересів, активності, самоорганізації та творчого мислення, забезпечення єдності
інтелектуального й особистісного розвитку. Інформатизація освіти й зростаючі
вимоги до якості та кількості висококваліфікованих фахівців створюють
необхідність розробки та впровадження інноваційних освітніх методик і
технологій.

Підвищення теоретичного рівня курсу фізики, інтеграція знань, новітні
засоби і форми навчання спонукають до подальших пошуків можливостей
застосування ЕОМ у навчальному процесі як стимулюючого та інтенсифікуючого
чинника.

Інформаційні технології навчання передбачають широке використання
комп'ютерної техніки та спеціального програмного забезпечення як потужного
засобу навчання фізики. Проблемами впровадження інформаційних технологій у
навчальний процес з фізики займалися: О. Бугайов, Є. Коршак, М. Головко, В.
Заболотний, Ю. Жук, О. Ляшенко, Н. Сосницька, М. Шут та ін. У працях [1--5] цих % Use -- for en-dash in ranges
вчених розглядаються питання удосконалення шкільного фізичного експерименту
засобами інформаційних технологій; поєднання традиційних засобів навчання,
зокрема підручника з фізики, з електронними; розробки ППЗ з вивчення окремих
тем курсу фізики. При роботі з ЕОМ створюється специфічний емоційний
настрій, формується алгоритмічна культура [2; 6].

Актуальність роботи та підстави для її виконання. При вивченні фізики
нерідко складається ситуація, коли учень на практиці не може застосувати набуті
знання, навіть у випадку їх осмисленого засвоєння. Отже, необхідно навчати
практичному використанню набутих знань й умінь [3; 4].

Реалізувати інтеграцію теоретичних знань та практичних навичок доцільно
шляхом залучення учнів до розв’язування дослідницько-творчих задач. Тому
актуальним є створення засобів, які за допомогою експериментальних задач
забезпечують оволодіння повноцінними вміннями. Завдання на розрахунок опору
електричного кола можна зустріти серед екзаменаційних та олімпіадних задач, а
також у завданнях зовнішнього незалежного оцінювання.

\textbf{Мета й завдання роботи.} Метою кваліфікаційної роботи є створення
програмного засобу для розв’язування задач на розрахунок опору електричного
кола. Для досягнення цієї мети поставлено такі завдання.
% Use the itemize environment for bulleted lists
\begin{itemize}
    \item Дослідити існуючі електронні засоби навчання фізики.
    \item Дослідити застосування різних способів розрахунку опору
    електричного кола.
    \item Розробити технічне завдання до продукту.
    \item Розробити інтерфейс та дизайн програмного продукту «Навчальна
    система для відображення та обчислення опору ділянки електричного
    кола».
\end{itemize}

Об'єкт, методи й засоби розроблення. Об’єктом розроблення програмного
засобу «Навчальна система для відображення та обчислення опору ділянки
електричного кола» є процес розв’язування задач на визначення опору
електричного кола за допомогою програмного засобу.

Розробленню програмного засобу передувало створення математичної
моделі задач на визначення опору електричного кола. Основу для цього склав
аналіз деяких типів з’єднання елементів та систематизація відповідних методів
обчислення опору.

Під час розроблення програмного продукту використана еволюційна
модель, заснована на таких принципах. Розробляється початкова версія продукту,
яка передається кінцевим користувачам для оцінки, після чого продукт
доробляється, враховуючи думку замовника. Аналогічно розробляються,
передаються й оцінюються проміжні версії програмного продукту, поки не
з’явиться повністю готовий продукт, який відповідає всім вимогам замовника.
Процеси специфікації, розробки та атестації програмного продукту ведуться
паралельно з постійним обміном інформації між ними.

В якості інструменту створення програмного засобу було обрано NetBeans
IDE 7.2.1 --- інтегроване середовище розробки (IDE) мовою програмування Java,
яке є безкоштовним, вільно поширюваним, з відкритим вихідним кодом.

Можливі сфери застосування. Програмний продукт «Навчальна система
для відображення та обчислення опору ділянки електричного кола» може
застосовуватися в навчальному процесі в шкільному курсі фізики під час
вивчення опору в електричному колі.

Взаємозв’язок з іншими роботами. Ця робота є частиною великого
спільного проєкту, що охоплював розроблення навальних засобів для різних
задач шкільної фізики, алгебри та хімії.

}
