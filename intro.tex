\chapter*{ВСТУП}
\phantomsection
\addcontentsline{toc}{chapter}{ВСТУП}

\textbf{Оцінка сучасного стану об’єкта розроблення.}
Сучасна криптографія перебуває у стані активного розвитку, що зумовлено як зростанням обсягів цифрової інформації, так і появою нових загроз інформаційній безпеці.
Більшість поширених криптосистем базуються на класичних алгебраїчних структурах — скінченних полях великих порядків, мультиплікативних групах простих модулів, або еліптичних кривих.
Водночас, зростає інтерес до альтернативних підходів, які використовують інші алгебраїчні об’єкти, зокрема скінченні кільця, та нетрадиційні джерела криптостійкості.
Особливо актуальним є пошук рішень, що забезпечують достатній рівень захисту при роботі з обмеженими обчислювальними ресурсами, наприклад, у пристроях Інтернету речей, вбудованих системах, смарт-картках.

\textbf{Актуальність роботи та підстави для її виконання.}
Більшість сучасних криптосистем вимагають складних обчислень над великими простими числами або полями високого порядку, що ускладнює їх застосування у ресурсно-обмежених середовищах.
Крім того, класичні припущення про складність (факторизація, дискретний логарифм) можуть бути поставлені під сумнів із розвитком квантових обчислень.
Тому актуальним є дослідження нових підходів до побудови симетричних криптосистем, стійкість яких ґрунтується на комбінаторній складності відображень та алгебраїчних перетворень, а не на стандартних задачах теорії чисел.
Особливий інтерес становлять системи, що дозволяють ефективно працювати з малими алгебраїчними структурами, забезпечуючи при цьому високий рівень захисту.

\textbf{Мета й завдання роботи.}
Метою кваліфікаційної роботи є розробка, теоретичне обґрунтування та програмна реалізація симетричної криптосистеми на основі сюр’єктивних відображень скінченних кілець та систем лінійних рівнянь.
Для досягнення цієї мети поставлено такі завдання:
\begin{itemize}
    \item Дослідити основи симетричної криптографії.
    \item Розробити алгоритми побудови скінченних кілець, ізоморфізмів та сюр’єктивних відображень між ними.
    \item Сформулювати та описати протокол обміну інформацією між двома сторонами з використанням систем лінійних рівнянь над кільцями.
    \item Провести теоретичний аналіз стійкості системи, оцінити складність криптоаналітичних атак.
    \item Реалізувати програмний прототип для генерації кілець, виконання основних операцій, шифрування та дешифрування.
    \item Дослідити можливості застосування системи у ресурсно-обмежених середовищах та перспективи інтеграції з технологіями доказів з нульовим розголошенням.
\end{itemize}

\textbf{Об’єкт, методи й засоби розроблення.}
Об’єктом дослідження є процес симетричного шифрування інформації на основі алгебраїчних структур скінченних кілець та систем лінійних рівнянь.
Методи дослідження включають теоретичний аналіз алгебраїчних властивостей кілець, комбінаторики відображень, побудову та дослідження алгоритмів, а також експериментальну перевірку працездатності на основі програмної реалізації.
У розробці використано сучасні засоби програмування (Rust, RustRover IDE для супутніх задач), а також стандартні бібліотеки для роботи з лінійною алгеброю та модульною арифметикою.

\textbf{Можливі сфери застосування.}
Розроблена криптосистема може бути використана для симетричного шифрування даних у випадках, коли сторони можуть безпечно узгодити спільний ключ, а також у пристроях з обмеженими ресурсами (IoT, смарт-картки).
Завдяки алгебраїчній структурі, система є перспективною для застосування у верифікованих обчисленнях із доказами з нульовим розголошенням, зокрема для доведення коректності операцій над зашифрованими даними.

\textbf{Взаємозв’язок з іншими роботами.}
Дана робота є логічним продовженням досліджень у галузі симетричної криптографії на основі нетрадиційних алгебраїчних структур, зокрема роботи~\cite{5}, та інтегрує сучасні підходи до побудови протоколів обміну інформацією, що враховують вимоги до ефективності, стійкості та можливості інтеграції з новітніми технологіями захисту даних.
