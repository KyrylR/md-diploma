\phantomsection
\chapter*{Вступ}
\markboth{Вступ}{Вступ}
\addcontentsline{toc}{chapter}{Вступ}



Цей приклад оформлення математичного тексту за допомогою системи \LaTeX, який може бути використаним для оформлення курсової або кваліфікаційної роботи, було розроблено Л.В.~Фардиголою, професором кафедри прикладної математики факультету математики і інформатики Харківського національного університету імені В.Н.~Каразіна.

\section*{Де взяти \LaTeX?}

\begin{center}
	\begin{tabular*}{\textwidth}{ll}
		\multicolumn{2}{c}{\large \LaTeX{} для \slshape Windows, macOS, Linux}\\
		\hline
		\color{darkblue} MikTeX &  \url{https://miktex.org/}   \\
		\color{darkblue} MacTeX & 
		\url{https://tug.org/mactex/}\\
		\color{darkblue} TeXLive &  \url{https://www.tug.org/texlive/} \\
		\hline
		\color{violet} TeXstudio & \color{blue} \url{https://www.texstudio.org/}  \\
		\color{violet} TeXmaker & \color{blue} \url{https://www.xm1math.net/texmaker/}\\ 
		\hline
		\\
		\multicolumn{2}{c}{\large \LaTeX{} \slshape online (без встановлення на комп'ютер)}
		\\
		\hline
		\color{darkgreen} Overleaf & \color{blue} \url{https://www.overleaf.com/} \\
		\hline
		\\
		\multicolumn{2}{c}{\large \LaTeX\,: \slshape довідкові ресурси}
		\\
		\hline
		\color{darkgreen} Overleaf & \color{blue} \url{https://www.overleaf.com/learn}\\
		\hline
		\color{khaki} WIKIBOOKS: \LaTeX & \color{blue}
		\url{https://en.wikibooks.org/wiki/LaTeX}\\
		\hline
	\end{tabular*}
\end{center}	

\section*{Структура документа} 

Цей документ набраний за допомогою системи \LaTeX. Він складається з головного файлу {\color{bluegreen}\verb|thesis.tex|}, який завантажує і монтує текст з файлів, які містять частини тексту. Розбиття тексту на частини не є обов'язковим, але це є дуже зручним, зокрема, допомагає в навігації в тексті і надає змогу працювати з лише частиною файлів, а, отже, і з відповідною частиною тексту. Крім головного файлу при побудові цього документу використовуються такі файли:
\begin{enumerate}[label={\upshape(\roman*)}, leftmargin=5.5ex, labelwidth=5.5ex] 
\item 
	{\color{bluegreen}\verb|titlepage*.tex|},
\item 
	{\color{bluegreen}\verb|abstracts.tex|},
\item
	{\color{bluegreen}\verb|intro.tex|},
\item
	{\color{bluegreen}\verb|chap-1.tex|},
\item
	{\color{bluegreen}\verb|chap-2.tex|},
\item
	{\color{bluegreen}\verb|chap-3.tex|},
\item
	{\color{bluegreen}\verb|conclusions.tex|},
\item 
	{\color{bluegreen}\verb|bibliography.tex|},
\item
	файли {\color{bluegreen}\verb|fig*.pdf|}, які містять рисунки (розташовані в теці {\color{bluegreen}\verb|pictures|}).
\end{enumerate}
Усі ці файли, а також допоміжний файл {\color{bluegreen}\verb|th-sty.sty|}, треба помістити в одну і ту саму  теку. Рисунки можна зібрати в окрему підтеку, і в цьому разі треба в преамбулі (тобто в тексті перед {\color{bluegreen}\verb|\begin{document}|}) явно вказати путь до цієї підтеки командою {\color{bluegreen}\verb|\graphicspath{ {<path>/} }|}, де замість {\color{bluegreen}\verb|<path>|} підставити відповідну путь, наприклад, в даному випадку рисунки зібрано у теці {\color{bluegreen}\verb|pictures|}, тому ця команда виглядає так:
{\color{bluegreen}\verb|\graphicspath{ {pictures/} }|} (див. преамбулу головного файлу {\color{bluegreen}\verb|thesis.tex|}). Іноді редактор, який ви використовуєте для набору та виправлення файлів, може самостійно визначати головний файл, але це відбувається не завжди, тому краще в налаштуваннях примусово зафіксувати цей файл на початку роботи.  У разі роботи на \url{overleaf.com} усі ці файли (з підтекою рисунків включно) треба запакувати в zip-архів і завантажити як новий проєкт на \url{overleaf.com}. 

Для правильного оформлення роботи треба вибрати титульну сторінку. Для студентів 4 курсу треба прибрати знак відсотку ({\color{bluegreen}\verb|%|}) перед командою {\color{bluegreen}\verb|\begin{titlepage}
	\setstretch{1}
	\begin{center}
		\Large Київський національний університет імені Тараса Шевченка\\
		Факультет комп'ютерних наук та кібернетики\\
		Кафедра прикладної математики
	\end{center}

	\vfill
		
	\begin{center}	
		\LARGE \bfseries Кваліфікаційна робота
		\\
			{\normalsize \bfseries \slshape бакалавра}
		\\[0.5\baselineskip]
		{\mdseries на тему} \bfseries\slshape <<Класи множин>> \\[0.5\baselineskip]
	\end{center}
	
	\vfill
	

	\setlength{\tabcolsep}{3pt}
	\hbox to \textwidth{\hfill\begin{tabular}{>{\slshape}rp{0.43\textwidth}}
			Виконав: &студент групи МП41, 
			
			спеціальність
			
			113 -- Прикладна математика,
			
			освітньо-професійна програма 
			
			<<Прикладна математика>>
			
			\textbf{Петренко~І.В.}
			\\[0.5\baselineskip]
			Науковий керівник: & доктор фіз.-мат. наук, 
			
			професор,
			
			професор кафедри 
			
			прикладної математики
			
			\textbf{Іваненко П.К.}	
			\\[0.5\baselineskip]
			Рецензент: & кандидат техн. наук, доцент,
			
			доцент кафедри 
			
			вищої математики
			
			Харківського національного 
			
			університету радіоелектроніки
			
			\textbf{Стеценко М.В.}
	\end{tabular}	}
	
\vspace{\baselineskip}
	
	\begin{center}
		Київ --- 2025 рік
	\end{center}
	
\end{titlepage}|} 
при цьому перед іншими двома командами мають стояти знаки відсотку ({\color{bluegreen}\verb|%|}). Для студентів 6 курсу ОПП треба прибрати знак відсотку ({\color{bluegreen}\verb|%|}) перед командою {\color{bluegreen}\verb|\begin{titlepage}

    \begin{center}
        \textbf{КИЇВСЬКИЙ НАЦІОНАЛЬНИЙ УНІВЕРСИТЕТ \\[-0.3em]
        ІМЕНІ ТАРАСА ШЕВЧЕНКА} \\
        \normalsize
        {\medium Факультет комп’ютерних наук та кібернетики \\[-0.3em]
        Кафедра інтелектуальних програмних систем}
    \end{center}

    \vspace{0.01cm}

    \begin{center}
    \textbf{Кваліфікаційна робота \\[-0.3em]
    на здобуття ступеня магістра} \\[-0.3em]
    за спеціальністю \textbf{121 Інженерія програмного забезпечення} \\[-0.3em]
    на тему: \\
    \textbf{ЗАСТОСУВАННЯ ІДЕЇ ОДНОРАЗОВОГО БЛОКНОТУ \\
    У КІЛЬЦЯХ ЛИШКІВ}
    \end{center}

    \vspace{0.5cm}

    \noindent
    \mediuml
    \begin{tabular}{@{}p{0.9\textwidth}@{}p{0.58\textwidth}@{}}
        Виконав студент 2-го курсу & \\[-0.3em]
        ОНП «Програмне забезпечення систем» & \\[-0.3em]
        Рябов Кирило Сергійович & \hspace{-1.7cm}\rule{2.5cm}{0.5pt} \\[-0.6em]
        & \hspace{-1.25cm} {\medium (підпис) } \\[0.2em]

        Науковий керівник: & \\[-0.3em]
        професор кафедри інтелектуальних програмних систем & \\[-0.3em]
        доктор фізико-математичних наук, професор & \\[-0.3em]
        Кривий Сергій Лук'янович & \hspace{-1.7cm}\rule{2.5cm}{0.6pt} \\[-0.6em]
        & \hspace{-1.25cm} {\medium (підпис) } \\[1.5em]
    \end{tabular}

    \noindent
    \mediuml
    \begin{tabular}{@{}p{0.35\textwidth}@{}p{0.7\textwidth}@{}}
        & Засвідчую, що в цій роботі \\[-0.3em]
        & немає запозичень з праць інших авторів \\[-0.3em]
        & без відповідних посилань. \\[-0.3em]
        & Студент \hspace{5.8cm}\rule{2.5cm}{0.6pt}\\[-0.6em]
        & \hspace{8.1cm} {\medium (підпис) } \\[1.5em]

        & Роботу розглянуто й допущено до захисту на \\[-0.3em]
        & засіданні кафедри інтелектуальних програмних \\
        & систем \\
        & «\rule{1.5cm}{0.4pt}» \rule{3cm}{0.4pt} 2025 р., \\
        & протокол № \rule{2cm}{0.4pt} \\
        & Завідувач кафедри \\
        & доктор фіз.-мат. наук, професор \\
        & Олександр ПРОВОТАР \hspace{2.55cm}\rule{2.5cm}{0.6pt}\\[-0.6em]
        & \hspace{8.1cm} {\medium (підпис) } \\
    \end{tabular}

    \vfill

    \begin{center}
        КИЇВ 2025
    \end{center}
\end{titlepage}
|} при цьому перед іншими двома командами мають стояти знаки відсотку ({\color{bluegreen}\verb|%|}). Для студентів 6 курсу ОНП треба прибрати знак відсотку ({\color{bluegreen}\verb|%|}) перед командою {\color{bluegreen}\verb|\begin{titlepage}
	\setstretch{1}
	\begin{center}
		\Large Київський національний університет імені Тараса Шевченка\\
		Факультет комп'ютерних наук та кібернетики\\
		Кафедра прикладної математики
	\end{center}

	\vfill
		
	\begin{center}	
		\LARGE \bfseries Кваліфікаційна робота
		\\
			{\normalsize \bfseries \slshape магістра}
		\\[0.5\baselineskip]
		{\mdseries на тему} \bfseries\slshape <<Класи множин>> \\[0.5\baselineskip]
	\end{center}
	
	\vfill
	

	\setlength{\tabcolsep}{3pt}
	\hbox to \textwidth{\hfill\begin{tabular}{>{\slshape}rp{0.42\textwidth}}
			Виконав: &студент групи МП61,
			
		спеціальність

	113 -- Прикладна математика,
	
	освітньо-наукова програма 
	
	<<Прикладна математика>>
	
	\textbf{Петренко~І.В.}
	\\[0.5\baselineskip]
	Науковий керівник: & доктор фіз.-мат. наук, 
	
	професор,
	
	професор кафедри 
	
	прикладної математики
	
	\textbf{Іваненко П.К.}	
	\\[0.5\baselineskip]
	Рецензент: & кандидат техн. наук, доцент,
	
	доцент кафедри 
	
	вищої математики
	
	Харківського національного 
	
	університету радіоелектроніки
	
	\textbf{Стеценко М.В.}
	\end{tabular}	}
	
\vspace{\baselineskip}
	
	\begin{center}
		Київ --- 2025 рік
	\end{center}
	
\end{titlepage}|} при цьому перед іншими двома командами мають стояти знаки відсотку ({\color{bluegreen}\verb|%|}). Свої дані (прізвища та ініціали автора (студента), наукового керівника та рецензента, а також пов'язану з ними інформацію) треба набрати в той файл, перед командою завантаження якого не стоїть знак відсотку ({\color{bluegreen}\verb|%|}), тобто у файл {\color{bluegreen}\verb|titlepage-BACHELOR.tex|} для студентів 4 курсу, у файл {\color{bluegreen}\verb|titlepage-OPP.tex|} для студентів 6 курсу ОПП,  у файл {\color{bluegreen}\verb|titlepage-ONP.tex|} для студентів 6 курсу ОНП.

Зверніть увагу на те, що зміст формується автоматично при правильному наборі заголовків частин тексту. Для заголовків не слід використовувати ручне форматування, наприклад, за допомогою команд центрування, масштабування та зміни шрифту. Ключовою вимогою правильної побудови документа є використання для заголовків спеціальних команд:
\begin{enumerate}[label={\upshape\arabic*)}, leftmargin=3.5ex, labelwidth=3.5ex]
\item	
{\color{bluegreen}\verb|\chapter{<text>}|}, 
\item
{\color{bluegreen}\verb|\section{<text>}|}, 
\item
{\color{bluegreen}\verb|\subsection{<text>}|}, 
\item
{\color{bluegreen}\verb|\subsubsection{<text>}|}, 
\item
{\color{bluegreen}\verb|\paragraph{<text>}|}.
\end{enumerate}
Кожна з наведених вище команд форматування заголовків має варіант із ``зірочкою'' (до прикладу, {\color{bluegreen}\verb|\section*{<text>}|}). У цьому разі такий підрозділ не отримає номера. У розділі, який Ви зараз читаєте  (тобто у вступі), було використано саме такі команди для заголовків. Дали в основному тексті використано варіанти команд без ``зірочки'' для форматування заголовків.

Для того,  щоб почати новий абзац, треба вставити хоча б один порожній рядок (кількість рядків ні на що не впливає), і в жодному разі не слід використовувати команду  {\color{bluegreen}\verb|\\|}. 

Для набору тверджень типу теорем, лем, означень тощо слід використовувати спеціальні оточення: {\color{bluegreen}\verb|{theorem}|} для теорем, {\color{bluegreen}\verb|{corollary}|} для наслідків,  {\color{bluegreen}\verb|{statement}|} для тверджень, {\color{bluegreen}\verb|{lemma}|} для лем, {\color{bluegreen}\verb|{definition}|} для означень, {\color{bluegreen}\verb|{problem}|} для задач, {\color{bluegreen}\verb|{remark}|} для зауважень, {\color{bluegreen}\verb|{example}|} для прикладів. До прикладу, набір теореми може виглядати так:
{\color{bluegreen}%
\begin{verbatim}
\begin{theorem}[Піфагора]
\label{pith}
Сума квадратів катетів дорівнює квадрату гіпотенузи.
\end{theorem}
\end{verbatim}%
}\noindent%
що в тексті має такий вигляд:
\begin{theorem}[Піфагора]
\label{pith}
Сума квадратів катетів дорівнює квадрату гіпотенузи.
\end{theorem} 
Текст в квадратних дужках і, власне, самі дужки не є обов'язковими і додаються за необхідності додати заголовок теореми. Інші приклади набору теорем, лем, наслідків тощо та використання перехресних посилань щодо них див. у розділах \ref{sec-1} і \ref{sec-2}.

Для посилання на всі нумеровані об'єкти використовуються команди {\color{bluegreen}\verb|\ref{<label>}|} і {\color{bluegreen}\verb|\eqref{<label>}|} (остання для посилання на формули), які разом з командою {\color{bluegreen}\verb|\label{<label>}|} генерують перехресні посилання, після клацання на які здійснюється перехід на цитований об'єкт. Тут  {\color{bluegreen}\verb|<label>|} є унікальною міткою (мітки не повинні повторюватися впродовж усього тексту), якою ми маркуємо кожен нумерований об'єкт і яку використовуємо для посилання на нього. Для створення міток використовуються довільні послідовності латинських літер і арабських цифр.
До прикладу, в теоремі Піфагора було створено мітку {\color{bluegreen}\verb|pith|} командою {\color{bluegreen}\verb|\label{pith}|}, тому для посилання на цю теорему ми використовуємо такий текст:

{\color{bluegreen}%
\begin{verbatim}
Розглянемо теорему Піфагора (див. теорему \ref{pith}).
\end{verbatim}%
}\noindent% 
що в тексті має такий вигляд:

\noindent%
Розглянемо теорему Піфагора (див. теорему \ref{pith}). 

Список літератури оформлюється за зразком, наведеним у файлі {\color{bluegreen}\verb|bibliography.tex|}. Кожний елемент цього списку повинен починатися командою {\color{bluegreen}\verb|\bibitem{<cite-label>}|}, де мітка {\color{bluegreen}\verb|<cite-label>|} є унікальною і не повинна збігатися ані з мітками інших елементів списку літератури, ані з мітками інших нумерованих об'єктів впродовж усього тексту. Для посилання на елемент цього списку використовується команда {\color{bluegreen}\verb|\cite{<cite-label>}|}. До прикладу, для посилання на підручник В.М. Радченка, який в списку літератури задається командою
{\color{bluegreen}\verb|\bibitem{VR}|},
використовуємо команду {\color{bluegreen}\verb|\cite{VR}|}, що в результаті дає такий текст: \cite{VR}.
Інші приклади використання перехресних посилань див. у розділах \ref{sec-1} і \ref{sec-2}.

Перед початком роботи в \LaTeX{} дуже бажано ознайомитися с основними принципами створення документів, наприклад, на \url{https://www.overleaf.com/learn} або \url{https://en.wikibooks.org/wiki/LaTeX}.  


