\chapter*{СКОРОЧЕННЯ ТА УМОВНІ ПОЗНАКИ}
\phantomsection
\addcontentsline{toc}{chapter}{СКОРОЧЕННЯ ТА УМОВНІ ПОЗНАКИ}

\noindent
AES --- Advanced Encryption Standard; \\
GEN-G --- алгоритм генерації визначального рядка кільця; \\
IoT --- Інтернет речей; \\
RSA --- криптосистема Rivest–Shamir–Adleman; \\
СЛР --- система лінійних рівнянь; \\
SLE --- system of liner equations (назва реалізації криптосистеми); \\
ZKP --- доказ з нульовим розголошенням (Zero-Knowledge Proof); \\
ZK-SNARK --- Zero-Knowledge Succinct Non-Interactive Argument of Knowledge; \\
ZK-STARK --- Zero-Knowledge Scalable Transparent Argument of Knowledge; \\
$Z_m$ --- кільце лишків за модулем $m$; \\
$G_m$, $G_k$ --- скінченні асоціативно-комутативні кільця з одиницею, ізоморфні $Z_m$, $Z_k$; \\
$\varphi$ --- ізоморфізм кілець; \\
$\psi, \lambda$ --- сюр'єктивні гомоморфізми кілець; \\
$\psi_1, \lambda_1$ --- бієкції фактор-множин; \\
$B_i, a_j$ --- матриці та вектори афінного перетворення; \\
$\bar{a}$ --- випадковий вектор для кожного блоку; \\
$k, m$ --- порядки відповідних кілець; \\
$p, q$ --- розмір блоку, кількість змінних у СЛР; \\
$r$ --- кількість афінних перетворень; \\
