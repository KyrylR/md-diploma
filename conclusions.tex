\chapter*{ВИСНОВКИ}
\phantomsection
\addcontentsline{toc}{chapter}{ВИСНОВКИ}

У роботі запропоновано новий підхід до побудови симетричної криптосистеми, що базується на сюр’єктивних відображеннях між скінченними асоціативно-комутативними кільцями з одиницею та системах лінійних рівнянь над кільцями лишків.

Основною перевагою розробленої системи є можливість використання алгебраїчних структур відносно невеликих порядків без необхідності складних обчислень над великими простими числами чи полями високого порядку.

Запропонований протокол забезпечує багаторівневу обфускацію даних, що підвищує стійкість до криптоаналітичних атак, зокрема перебірних і статистичних.

Стійкість системи ґрунтується на комбінаторній складності множини відображень та ізоморфізмів між кільцями, а також на ймовірнісному характері шифрування, що унеможливлює частотний аналіз.

Практична реалізація алгоритмів у вигляді програмної бібліотеки підтвердила працездатність і ефективність основних ідей.

Розроблена система може бути впроваджена для симетричного шифрування даних у випадках, коли сторони можуть безпечно узгодити спільний ключ, а також у пристроях з обмеженими ресурсами (наприклад, IoT, смарт-картки).

Алгебраїчна структура протоколу створює перспективи для застосування у верифікованих обчисленнях та протоколах з нульовим розголошенням.

Наукова та практична значущість роботи полягає у розвитку напрямів симетричної криптографії на основі нових джерел стійкості, а також у потенціалі для подальшої інтеграції з сучасними технологіями захисту даних.

Подальші дослідження доцільно спрямувати на поглиблений криптоаналіз, оптимізацію продуктивності, розробку рекомендацій щодо вибору параметрів, а також на розширення сфери застосування, зокрема у протоколах верифікованого шифрування та захищених обчислень.