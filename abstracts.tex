\begin{center}
    \textbf{РЕФЕРАТ}
\end{center}

Обсяг роботи 63 сторінки, 8 ілюстрацій, 5 таблиць, 27 джерел.

СКІНЧЕННІ КІЛЬЦЯ, СИМЕТРИЧНА КРИПТОСИСТЕМА, СЮР’ЄКТИВНІ ВІДОБРАЖЕННЯ, СИСТЕМИ ЛІНІЙНИХ РІВНЯНЬ, ІЗОМОРФІЗМ КІЛЕЦЬ,
КОМБІНАТОРНА КРИПТОСТІЙКІСТЬ, ПРОГРАМНА РЕАЛІЗАЦІЯ, ВЕРИФІКОВАНІ ОБЧИСЛЕННЯ.

Об’єктом дослідження є процес симетричного шифрування інформації на основі алгебраїчних структур скінченних кілець та систем лінійних рівнянь.

Предметом роботи є розробка симетричної криптосистеми, що використовує сюр’єктивні відображення між кільцями та афінні
перетворення для забезпечення захисту даних.

Метою роботи є створення, теоретичне обґрунтування та програмна реалізація симетричної криптосистеми, що використовує
сюр'єктивні відображення скінченних кілець та їх ізоморфізми з використанням систем лінійних рівнянь над кільцями лишків.

У роботі застосовано методи теоретичного аналізу алгебраїчних структур, комбінаторики відображень, побудови та дослідження алгоритмів.
Програмна реалізація виконана мовою Rust із використанням стандартних бібліотек для лінійної алгебри та модульної арифметики,
а також інтегрованого середовища RustRover IDE для реалізації бібліотеки.

У результаті розроблено алгоритм GEN-G для генерації скінченних кілець та ізоморфізмів, описано та реалізовано протокол
симетричного шифрування/дешифрування з багаторівневою обфускацією даних, проведено теоретичний аналіз стійкості, створено
програмну бібліотеку для генерації кілець, виконання основних операцій, шифрування та дешифрування.
Досліджено можливості застосування системи у верифікованих обчисленнях із доказами з нульовим розголошенням.

Розроблений програмний продукт може бути впроваджений для симетричного шифрування даних у випадках, коли сторони можуть
безпечно узгодити спільний ключ, а також у пристроях з обмеженими ресурсами (IoT, смарт-картки).

Результати роботи можуть бути використані для подальшого розвитку симетричних криптосистем на основі алгебраїчних структур,
а також для інтеграції з сучасними протоколами захисту інформації.
Доцільним є продовження досліджень щодо оптимізації продуктивності, розширення сфери застосування та впровадження в існуючі системи.