\begin{center}
{\Large \textbf{РЕФЕРАТ}}
\end{center}

\vspace{1cm}

{\Large

Обсяг роботи 57 сторінок, 12 ілюстрацій, 4 таблиці, 26 джерел посилань.

\begin{sloppypar}
ВІДОБРАЖЕННЯ ДІЛЯНКИ ЕЛЕКТРИЧНОГО КОЛА, ЕЛЕКТРИЧНЕ КОЛО, ІНТЕРФЕЙС ПРОГРАМНОГО ПРОДУКТУ, ІНФОРМАЦІЙНІ ТЕХНОЛОГІЇ НАВЧАННЯ, НАВЧАЛЬНА СИСТЕМА, РОЗРАХУНОК ОПОРУ.
\end{sloppypar}

Об’єктом роботи є процес розв’язування задач на визначення опору електричного кола за допомогою програмного засобу «Навчальна система для відображення та обчислення опору ділянки електричного кола». Предметом
роботи є програмний засіб для розвязування розрахункових задач на визначення
опору електричного кола.

Метою роботи є створення навчального програмного засобу для
розвязування задач на розрахунок опору електричного кола.

Методи розроблення: комп’ютерне моделювання, методи обчислення опору
ділянок електричних кіл, розробка програмного продукту на основі еволюційної
моделі.

Інструменти розроблення: безкоштовне, вільно поширюване інтегроване
середовище розробки NetBeans IDE 7.2.1, мова програмування Java.

Результати роботи: виконано загальний огляд електронних засобів навчання
фізики, проаналізовано переваги та недоліки використання електронних засобів у
процесі навчання, розроблено програмний продукт «Навчальна система для
відображення та обчислення опору ділянки електричного кола», який дозволяє
наочно демонструвати процеси побудови ділянки електричного кола та
обчислення її опору.

За методами розробки та інструментальними засобами робота виконувалася
у комплексі з роботами з розв’язання задач шкільної алгебри та хімії.
Програмний продукт «Навчальна система для відображення та обчислення
опору ділянки електричного кола» може застосовуватися в навчальному процесі в
шкільному курсі фізики під час вивчення опору у електричному колі.

}
