\clearpage
\phantomsection
\addcontentsline{toc}{chapter}{ПЕРЕЛІК ДЖЕРЕЛ ПОСИЛАННЯ}

\begin{thebibliography}{XX}

    \bibitem{9}
    Kerckhoffs A. La cryptographie militaire. // \textit{Journal des sciences militaires}. 1883. Т. IX. С. 5--38, 161--191.

    \bibitem{8}
    Shannon C. E. Communication Theory of Secrecy Systems. // \textit{Bell System Technical Journal}. 1949. Т. 28(4). С. 656--715.

    \bibitem{1}
    Mao W. \textit{Modern Cryptography}. New Jersey : Pearson Education, Prentice Hall Professional Technical Reference, 2004. 768 с.

    \bibitem{4}
    Berczes A., Lajos H., Hirete-Kohno N., Kovacs T. A key exchange protocol based on Diophantine equations and S-integers. // \textit{JSIAM Letters}. 2014. С. 85--88.

    \bibitem{VernamWiki}
    Шифр Вернама // Вікіпедія : вільна енциклопедія. URL: \url{https://uk.wikipedia.org/wiki/%D0%A8%D0%B8%D1%84%D1%80_%D0%92%D0%B5%D1%80%D0%BD%D0%B0%D0%BC%D0%B0} (дата звернення: 28.04.2025).

    \bibitem{5}
    Kryvyi S. та ін. Symmetric system for Exchange Information on the Base of Surjective Isomorphism of Rings. // \textit{12th Int. IEEE Conf. on Dependable Systems, Services and Technologies (DESSERT 2022)} (Kyiv, Ukraine, December 9-11, 2022). 2022. С. 1--7.

    \bibitem{6}
    Shoup V. \textit{An Computational Introduction to Number Theory and Algebra}. Cambridge University Press, 2008. 580 с.

    \bibitem{2}
    Kameswari P. A., Sriniasarao S. S., Belay A. An application of Linear Diophantine equations to Cryptography. // \textit{Advanced in Mathematics: Scientific Journal}. 2021. Т. 10. С. 2799--2806.

    \bibitem{7}
    Кривий С. Л. \textit{Лінійні Діофантові обмеження та їх застосування}. Київ : Інтерсервіс, 2021. 257 с.

    \bibitem{12}
    Boneh D., Shoup V. \textit{A Graduate Course in Applied Cryptography}. Проєкт, 2020. URL: \url{https://crypto.stanford.edu/~dabo/cryptobook/} (дата звернення: YYYY-MM-DD).

    \bibitem{13}
    Goldreich O. \textit{Foundations of Cryptography. Vol. 2: Basic Applications}. Cambridge University Press, 2004. 423 с.

    \bibitem{14}
    Groth J. On the Size of Pairing-Based Non-interactive Arguments. // \textit{EUROCRYPT 2016}. LNCS, т. 9666. Springer, 2016. С. 305--326.

    \bibitem{15}
    Ben-Sasson E. та ін. SNARKs for C: Verifying Program Executions Succinctly and in Zero Knowledge. // \textit{CRYPTO 2013}. LNCS, т. 8042. Springer, 2013. С. 90--108.

    \bibitem{16}
    Bünz B. та ін. Bulletproofs: Short Proofs for Confidential Transactions and More. // \textit{IEEE Symposium on Security and Privacy (SP)}. 2018. С. 315--334.

    \bibitem{20}
    Bellare M., Rogaway P. Introduction to Modern Cryptography. Конспект лекцій курсу UCSD CSE 207. 2005.

    \bibitem{21}
    Katz J., Lindell Y. \textit{Introduction to Modern Cryptography}. 2-ге вид. CRC Press, 2014. 603 с.

    \bibitem{22}
    Stinson D. R., Paterson M. B. \textit{Cryptography: Theory and Practice}. 4-те вид. CRC Press, 2018. 598 с.

    \bibitem{24}
    Diffie W., Hellman M. E. New Directions in Cryptography. // \textit{IEEE Transactions on Information Theory}. 1976. Т. 22(6). С. 644--654.

    \bibitem{25}
    ChaCha20 and Poly1305 for IETF Protocols. RFC 8439. 2018.

    \bibitem{sle_ref}
    KyrylR. \textit{SLE Cryptosystem Reference Implementation}. GitHub, 2024. \url{https://github.com/KyrylR/sle-cryptosystem-wasm}

    \bibitem{zkp_origin}
    Goldwasser S., Micali S., Rackoff C. The knowledge complexity of interactive proof systems. // \textit{SIAM Journal on computing}. 1989. Т. 18(1). С. 186--208.

    \bibitem{pinocchio}
    Parno B., Howell J., Gentry C., Raykova M. Pinocchio: Nearly Practical Verifiable Computation. // \textit{IEEE Symposium on Security and Privacy (SP)}. 2013. С. 238--252.

    \bibitem{stark}
    Ben-Sasson E., Bentov I., Horesh Y., Riabzev M. Scalable, transparent, and post-quantum secure computational integrity. // \textit{IACR Cryptology ePrint Archive}. 2018. URL: \url{https://eprint.iacr.org/2018/046}

    \bibitem{elgamal}
    ElGamal T. A public key cryptosystem and a signature scheme based on discrete logarithms. // \textit{IEEE Transactions on Information Theory}. 1985. Т. 31(4). С. 469--472.

    \bibitem{nist_gcm}
    Dworkin M. Recommendation for Block Cipher Modes of Operation: Galois/Counter Mode (GCM) and GMAC. // \textit{NIST Special Publication 800-38D}. 2007.

    \bibitem{rustcrypto_aes}
    RustCrypto Developers. AES-GCM crate documentation. URL: \url{https://docs.rs/aes-gcm/} (дата звернення: YYYY-MM-DD).

    \bibitem{rustcrypto_chacha}
    RustCrypto Developers. ChaCha20Poly1305 crate documentation. URL: \url{https://docs.rs/chacha20poly1305/} (дата звернення: YYYY-MM-DD).

\end{thebibliography}
