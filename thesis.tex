%%%%%%%%%%%%%%%%%%%%%%%%%%%%%%%%%%%%
%Author: L.V. Fardigola (fardigola@karazin.ua), 
%Dept. of Applied Mathematics,
%V.N. Karazin Kharkiv National University, Kharkiv, Ukraine
%%%%%%%%%%%%%%%%%%%%%%%%%%%%%%%%%%%%%
\documentclass[14pt,a4paper,oneside]{book}
\usepackage{cmap}
\usepackage{amsthm,amsfonts,mathtools,amscd,amssymb, mathrsfs}
\usepackage{latexsym}
\usepackage{euscript}
\usepackage{enumitem}
\usepackage{multicol}
\usepackage{bbm}
%\usepackage{rotating}
\usepackage{graphicx}
%\usepackage{epstopdf}
\usepackage{tikz}
\usepackage{array}
%\usepackage{pgfplots}
\usepackage{caption}
\usepackage{subcaption}
\usepackage{cite}
\usepackage{listings}
\usepackage{setspace}
\usepackage{times}
\usepackage{microtype}
%%%%%%%%%%%%%%%%%%%% languages %%%%%%%%%%%
\usepackage{fontspec}
\setmainfont{Times New Roman}
\setmonofont{DejaVu Sans Mono}
%%%%%%%%%%%%%%%%%%%%%%%%%%%%%%%%%%%%%%%%%%
\usepackage{xcolor}
\usepackage[
    unicode=true,     % Enable unicode support
    colorlinks=true,  % Enable link coloring
    linkcolor=black,  % Set internal links (like ToC entries) to black
    urlcolor=black,    % Optional: Keep URLs blue (or set to black)
    citecolor=green   % Optional: Keep citations green (or set to black)
]{hyperref}           % Load hyperref (usually last or near last)
%%%%%%%%%%%%%%%
\usepackage[left=24mm, right=15mm, top=20mm, bottom=20mm]{geometry}
\tolerance=9000
%\textwidth=165mm
%\textheight=250mm
%\oddsidemargin=4.6mm
%\voffset=-20mm
\binoppenalty=1 \relpenalty=1
\allowdisplaybreaks
%\everymath{\displaystyle}
%\renewcommand{\baselinestretch}{1.44}
\onehalfspacing
%%%%%%%%%%%%%%%%%%%%%%%%%%%
\usepackage{th-sty}
%%%%%%%%%%%%%%%%%%%%%%%%%
\usepackage{fancyhdr}
\pagestyle{fancy}
\fancyhf{}
\renewcommand{\headrulewidth}{0pt}
\renewcommand{\footrulewidth}{0pt}
\fancyfoot[C]{\thepage}
\fancypagestyle{plain}{
  \fancyhf{}
  \renewcommand{\headrulewidth}{0pt}
  \renewcommand{\footrulewidth}{0pt}
  \fancyfoot[C]{\thepage}
}
%%%%%%%%%%%%%%%%%%%%%%%%%
\graphicspath{ {pictures/} }
%%%%%%%%%%%%%%%%%%%%%%%%%%%%%%%%%%%%
\theoremstyle{dplplain}
\newtheorem{theorem}{Теорема}[chapter]
\newtheorem{corollary}[theorem]{Наслідок}%[chapter]
\newtheorem{statement}[theorem]{Твердження}%[chapter]
\newtheorem{lemma}[theorem]{Лема}%[chapter]
\theoremstyle{dpldefinition}
\newtheorem{definition}[theorem]{Означення}%[chapter]
\theoremstyle{dplremark}
\newtheorem{problem}[theorem]{Задача}%[chapter]
\newtheorem{remark}[theorem]{Зауваження}%[chapter]
\newtheorem{example}[theorem]{Приклад}%[chapter]
%%%%%%%%%%%%%%%%%%%%%%%%%%%%%%%%%%%%%%%
\newcommand{\R}{{\mathbb R}}
\newcommand{\RR}{\overline{\mathbb R}}
\newcommand{\N}{{\mathbb N}}
\newcommand{\Z}{{\mathbb Z}}
\renewcommand{\C}{{\mathbb C}}
\newcommand{\Q}{{\mathbb Q}}
%%%%%%%%%%%%%%%%%%%%%%%%%%%%%%%%%%%%%%%%%%%%%%%%%%%%%%%%%%
\newcommand{\fnnl}{\left[\!\left[\!\left|}
\newcommand{\fnnr}{\right|\!\right]\!\right]}
\newcommand{\fnl}{\left[\!\left|}
\newcommand{\fnr}{\right|\!\right]}
\newcommand{\nnnl}{\left|\!\left|\!\left|}
\newcommand{\nnnr}{\right|\!\right|\!\right|}
\newcommand{\nnl}{\left\|}
\newcommand{\nnr}{\right\|}
\newcommand{\nl}{\left|}
\newcommand{\nr}{\right|}
\newcommand{\pl}{\left(}
\newcommand{\pr}{\right)}
\newcommand{\bl}{\left[}
\newcommand{\br}{\right]}
\newcommand{\bbl}{\left\{}
\newcommand{\bbr}{\right\}}
\newcommand{\pnl}{\left(\kern-0.21em\left|}
\newcommand{\pnr}{\right|\kern-0.21em\right)}
\newcommand{\lal}{\left\langle}
\newcommand{\lar}{\right\rangle}
%%%%%%%%%%%%%%%%%%%%%%%%%%%%%%%%%%%%%%%%%%%%%%%%%%%%%%%%%%
\newcommand{\supp}{\mathop{\mathrm{supp}}}
\newcommand{\sign}{\mathop{\mathrm{sign}}}
\newcommand{\sinc}{\mathop{\mathrm{sinc}}}
\newcommand{\Ker}{\mathop{\mathrm{Ker}}}
\newcommand{\lin}{\mathop{\mathrm{lin}}}
\newcommand{\sgn}{\mathop{\mathrm{sgn}}}
\newcommand{\tr}{\mathop{\mathrm{tr}}}
\renewcommand{\Im}{\mathop{\mathrm{Im}}}
\renewcommand{\Re}{\mathop{\mathrm {Re}}}
\newcommand{\K}{\mathcal K}
\newcommand{\HH}{\mathcal H}
\newcommand{\m}{\mathsf{m}}
\newcommand{\mm}{\operatorname{\mathsf{mod}}}
\newcommand{\PP}{\mathcal P}
\newcommand{\PPP}{\overline{\mathcal P}}
\newcommand{\SSS}{\mathcal S}
\newcommand{\FF}{\mathcal F}
\newcommand{\LL}{\mathcal L}
\newcommand{\BB}{\mathcal B}
\newcommand{\one}{\mathbbmss 1}
\newcommand{\ubar}[1]{\text{\b{$#1$}}{}}
\newcommand{\V}{\mathbb V}
\newcommand{\BV}{\mathbb{BV}}
\newcommand{\AC}{\mathbb{AC}}
\newcommand{\midd}{\,\middle|\,}
%%%%%%%%%%%%%%%%%%%%%%%%
\DeclareMathOperator*{\esssup}{ess\,sup}
\DeclareMathOperator*{\essinf}{ess\,inf}
\DeclareMathOperator*{\bigtimes}{\mbox{\raisebox{-0.5ex}{\huge$\times$}}}
\DeclareMathOperator*{\smalltimes}{\mbox{\raisebox{-0.3ex}{\large$\times$}}}
\DeclareMathOperator*{\ootimes}{\overline{\otimes}}
%%%%%%%%%%%% LISTING %%%%%%%%%%%%%%%%%%%
\lstset{language=Python}
\lstset{frame=lines}
\lstset{basicstyle=\normalsize}
\lstset{prebreak=\raisebox{-2ex}[0ex][0ex]{\ensuremath{\hookleftarrow}}}
\lstset{breaklines=true}
\captionsetup[lstlisting]{font=large}

\begin{document}

\begin{titlepage}
	\setstretch{1}
	\begin{center}
		\Large Київський національний університет імені Тараса Шевченка\\
		Факультет комп'ютерних наук та кібернетики\\
		Кафедра прикладної математики
	\end{center}

	\vfill
		
	\begin{center}	
		\LARGE \bfseries Кваліфікаційна робота
		\\
			{\normalsize \bfseries \slshape магістра}
		\\[0.5\baselineskip]
		{\mdseries на тему} \bfseries\slshape <<Класи множин>> \\[0.5\baselineskip]
	\end{center}
	
	\vfill
	

	\setlength{\tabcolsep}{3pt}
	\hbox to \textwidth{\hfill\begin{tabular}{>{\slshape}rp{0.42\textwidth}}
			Виконав: &студент групи МП62,
			
		спеціальність
	
	113 -- Прикладна математика,
	
	освітньо-професійна програма 
	
	<<Прикладна математика>>
	
	\textbf{Петренко~І.В.}
	\\[0.5\baselineskip]
	Науковий керівник: & доктор фіз.-мат. наук, 
	
	професор,
	
	професор кафедри 
	
	прикладної математики
	
	\textbf{Іваненко П.К.}	
	\\[0.5\baselineskip]
	Рецензент: & кандидат техн. наук, доцент,
	
	доцент кафедри 
	
	вищої математики
	
	Харківського національного 
	
	університету радіоелектроніки
	
	\textbf{Стеценко М.В.}
	\end{tabular}	}
	
\vspace{\baselineskip}
	
	\begin{center}
		Київ --- 2025 рік
	\end{center}
	
\end{titlepage}

\refstepcounter{page}

\chapter*{Анотації}


\subsubsection{Петренко Іван Васильович. Класи множин.}
Будь ласка, помістіть тут текст вашої анотації українською. Текст анотації не повинен містити формул або посилань на текст роботи (тобто бути повністю автономним). Це потрібно для того, щоб анотацію можна було завантажити в інтернет і читати окремо від роботи.

\textbf{Ключові слова:} множина, кільце, алгебра, сигма-алгебра, борельова множина




\selectlanguage{english} 
%перемикання мови для введення тексту англійською

\subsubsection{Petrenko Ivan. Classes of Sets.}
	Please, put the  text of your abstract in English here. The text of the abstract should not contain formulae or references (i.e. it should be completely autonomous). This is necessary for loading it on the Internet and for reading it separately from the work.

\textbf{Keywords:} set, ring, algebra, sigma-algebra, Borel set.


\selectlanguage{ukrainian}
%перемикання мови для введення тексту українською

\renewcommand{\contentsname}{ЗМІСТ}
\renewcommand{\cfttoctitlefont}{\centering\Large\bfseries}
\renewcommand{\aftertoctitle}{\\[1.5em]}

\renewcommand{\cftchapleader}{\cftdotfill{\cftdotsep}}
\tableofcontents

\chapter*{СКОРОЧЕННЯ ТА УМОВНІ ПОЗНАКИ}
\phantomsection
\addcontentsline{toc}{chapter}{СКОРОЧЕННЯ ТА УМОВНІ ПОЗНАКИ}

{\Large

\noindent БЗ --- база знань; \\
БНФ --- Бекуса-Наура форма; \\
ВІС --- виконавча інформаційна система; \\
ВП --- віртуальне підприємство;

}


\phantomsection
\chapter*{Вступ}
\markboth{Вступ}{Вступ}
\addcontentsline{toc}{chapter}{Вступ}



Цей приклад оформлення математичного тексту за допомогою системи \LaTeX, який може бути використаним для оформлення курсової або кваліфікаційної роботи, було розроблено Л.В.~Фардиголою, професором кафедри прикладної математики факультету математики і інформатики Харківського національного університету імені В.Н.~Каразіна.

\section*{Де взяти \LaTeX?}

\begin{center}
	\begin{tabular*}{\textwidth}{ll}
		\multicolumn{2}{c}{\large \LaTeX{} для \slshape Windows, macOS, Linux}\\
		\hline
		\color{darkblue} MikTeX &  \url{https://miktex.org/}   \\
		\color{darkblue} MacTeX & 
		\url{https://tug.org/mactex/}\\
		\color{darkblue} TeXLive &  \url{https://www.tug.org/texlive/} \\
		\hline
		\color{violet} TeXstudio & \color{blue} \url{https://www.texstudio.org/}  \\
		\color{violet} TeXmaker & \color{blue} \url{https://www.xm1math.net/texmaker/}\\ 
		\hline
		\\
		\multicolumn{2}{c}{\large \LaTeX{} \slshape online (без встановлення на комп'ютер)}
		\\
		\hline
		\color{darkgreen} Overleaf & \color{blue} \url{https://www.overleaf.com/} \\
		\hline
		\\
		\multicolumn{2}{c}{\large \LaTeX\,: \slshape довідкові ресурси}
		\\
		\hline
		\color{darkgreen} Overleaf & \color{blue} \url{https://www.overleaf.com/learn}\\
		\hline
		\color{khaki} WIKIBOOKS: \LaTeX & \color{blue}
		\url{https://en.wikibooks.org/wiki/LaTeX}\\
		\hline
	\end{tabular*}
\end{center}	

\section*{Структура документа} 

Цей документ набраний за допомогою системи \LaTeX. Він складається з головного файлу {\color{bluegreen}\verb|thesis.tex|}, який завантажує і монтує текст з файлів, які містять частини тексту. Розбиття тексту на частини не є обов'язковим, але це є дуже зручним, зокрема, допомагає в навігації в тексті і надає змогу працювати з лише частиною файлів, а, отже, і з відповідною частиною тексту. Крім головного файлу при побудові цього документу використовуються такі файли:
\begin{enumerate}[label={\upshape(\roman*)}, leftmargin=5.5ex, labelwidth=5.5ex] 
\item 
	{\color{bluegreen}\verb|titlepage*.tex|},
\item 
	{\color{bluegreen}\verb|abstracts.tex|},
\item
	{\color{bluegreen}\verb|intro.tex|},
\item
	{\color{bluegreen}\verb|chap-1.tex|},
\item
	{\color{bluegreen}\verb|chap-2.tex|},
\item
	{\color{bluegreen}\verb|chap-3.tex|},
\item
	{\color{bluegreen}\verb|conclusions.tex|},
\item 
	{\color{bluegreen}\verb|bibliography.tex|},
\item
	файли {\color{bluegreen}\verb|fig*.pdf|}, які містять рисунки (розташовані в теці {\color{bluegreen}\verb|pictures|}).
\end{enumerate}
Усі ці файли, а також допоміжний файл {\color{bluegreen}\verb|th-sty.sty|}, треба помістити в одну і ту саму  теку. Рисунки можна зібрати в окрему підтеку, і в цьому разі треба в преамбулі (тобто в тексті перед {\color{bluegreen}\verb|\begin{document}|}) явно вказати путь до цієї підтеки командою {\color{bluegreen}\verb|\graphicspath{ {<path>/} }|}, де замість {\color{bluegreen}\verb|<path>|} підставити відповідну путь, наприклад, в даному випадку рисунки зібрано у теці {\color{bluegreen}\verb|pictures|}, тому ця команда виглядає так:
{\color{bluegreen}\verb|\graphicspath{ {pictures/} }|} (див. преамбулу головного файлу {\color{bluegreen}\verb|thesis.tex|}). Іноді редактор, який ви використовуєте для набору та виправлення файлів, може самостійно визначати головний файл, але це відбувається не завжди, тому краще в налаштуваннях примусово зафіксувати цей файл на початку роботи.  У разі роботи на \url{overleaf.com} усі ці файли (з підтекою рисунків включно) треба запакувати в zip-архів і завантажити як новий проєкт на \url{overleaf.com}. 

Для правильного оформлення роботи треба вибрати титульну сторінку. Для студентів 4 курсу треба прибрати знак відсотку ({\color{bluegreen}\verb|%|}) перед командою {\color{bluegreen}\verb|\input{titlepage-BACHELOR}|} 
при цьому перед іншими двома командами мають стояти знаки відсотку ({\color{bluegreen}\verb|%|}). Для студентів 6 курсу ОПП треба прибрати знак відсотку ({\color{bluegreen}\verb|%|}) перед командою {\color{bluegreen}\verb|\begin{titlepage}
	\setstretch{1}
	\begin{center}
		\Large Київський національний університет імені Тараса Шевченка\\
		Факультет комп'ютерних наук та кібернетики\\
		Кафедра прикладної математики
	\end{center}

	\vfill
		
	\begin{center}	
		\LARGE \bfseries Кваліфікаційна робота
		\\
			{\normalsize \bfseries \slshape магістра}
		\\[0.5\baselineskip]
		{\mdseries на тему} \bfseries\slshape <<Класи множин>> \\[0.5\baselineskip]
	\end{center}
	
	\vfill
	

	\setlength{\tabcolsep}{3pt}
	\hbox to \textwidth{\hfill\begin{tabular}{>{\slshape}rp{0.42\textwidth}}
			Виконав: &студент групи МП62,
			
		спеціальність
	
	113 -- Прикладна математика,
	
	освітньо-професійна програма 
	
	<<Прикладна математика>>
	
	\textbf{Петренко~І.В.}
	\\[0.5\baselineskip]
	Науковий керівник: & доктор фіз.-мат. наук, 
	
	професор,
	
	професор кафедри 
	
	прикладної математики
	
	\textbf{Іваненко П.К.}	
	\\[0.5\baselineskip]
	Рецензент: & кандидат техн. наук, доцент,
	
	доцент кафедри 
	
	вищої математики
	
	Харківського національного 
	
	університету радіоелектроніки
	
	\textbf{Стеценко М.В.}
	\end{tabular}	}
	
\vspace{\baselineskip}
	
	\begin{center}
		Київ --- 2025 рік
	\end{center}
	
\end{titlepage}|} при цьому перед іншими двома командами мають стояти знаки відсотку ({\color{bluegreen}\verb|%|}). Для студентів 6 курсу ОНП треба прибрати знак відсотку ({\color{bluegreen}\verb|%|}) перед командою {\color{bluegreen}\verb|\input{titlepage-ONP}|} при цьому перед іншими двома командами мають стояти знаки відсотку ({\color{bluegreen}\verb|%|}). Свої дані (прізвища та ініціали автора (студента), наукового керівника та рецензента, а також пов'язану з ними інформацію) треба набрати в той файл, перед командою завантаження якого не стоїть знак відсотку ({\color{bluegreen}\verb|%|}), тобто у файл {\color{bluegreen}\verb|titlepage-BACHELOR.tex|} для студентів 4 курсу, у файл {\color{bluegreen}\verb|titlepage-OPP.tex|} для студентів 6 курсу ОПП,  у файл {\color{bluegreen}\verb|titlepage-ONP.tex|} для студентів 6 курсу ОНП.

Зверніть увагу на те, що зміст формується автоматично при правильному наборі заголовків частин тексту. Для заголовків не слід використовувати ручне форматування, наприклад, за допомогою команд центрування, масштабування та зміни шрифту. Ключовою вимогою правильної побудови документа є використання для заголовків спеціальних команд:
\begin{enumerate}[label={\upshape\arabic*)}, leftmargin=3.5ex, labelwidth=3.5ex]
\item	
{\color{bluegreen}\verb|\chapter{<text>}|}, 
\item
{\color{bluegreen}\verb|\section{<text>}|}, 
\item
{\color{bluegreen}\verb|\subsection{<text>}|}, 
\item
{\color{bluegreen}\verb|\subsubsection{<text>}|}, 
\item
{\color{bluegreen}\verb|\paragraph{<text>}|}.
\end{enumerate}
Кожна з наведених вище команд форматування заголовків має варіант із ``зірочкою'' (до прикладу, {\color{bluegreen}\verb|\section*{<text>}|}). У цьому разі такий підрозділ не отримає номера. У розділі, який Ви зараз читаєте  (тобто у вступі), було використано саме такі команди для заголовків. Дали в основному тексті використано варіанти команд без ``зірочки'' для форматування заголовків.

Для того,  щоб почати новий абзац, треба вставити хоча б один порожній рядок (кількість рядків ні на що не впливає), і в жодному разі не слід використовувати команду  {\color{bluegreen}\verb|\\|}. 

Для набору тверджень типу теорем, лем, означень тощо слід використовувати спеціальні оточення: {\color{bluegreen}\verb|{theorem}|} для теорем, {\color{bluegreen}\verb|{corollary}|} для наслідків,  {\color{bluegreen}\verb|{statement}|} для тверджень, {\color{bluegreen}\verb|{lemma}|} для лем, {\color{bluegreen}\verb|{definition}|} для означень, {\color{bluegreen}\verb|{problem}|} для задач, {\color{bluegreen}\verb|{remark}|} для зауважень, {\color{bluegreen}\verb|{example}|} для прикладів. До прикладу, набір теореми може виглядати так:
{\color{bluegreen}%
\begin{verbatim}
\begin{theorem}[Піфагора]
\label{pith}
Сума квадратів катетів дорівнює квадрату гіпотенузи.
\end{theorem}
\end{verbatim}%
}\noindent%
що в тексті має такий вигляд:
\begin{theorem}[Піфагора]
\label{pith}
Сума квадратів катетів дорівнює квадрату гіпотенузи.
\end{theorem} 
Текст в квадратних дужках і, власне, самі дужки не є обов'язковими і додаються за необхідності додати заголовок теореми. Інші приклади набору теорем, лем, наслідків тощо та використання перехресних посилань щодо них див. у розділах \ref{sec-1} і \ref{sec-2}.

Для посилання на всі нумеровані об'єкти використовуються команди {\color{bluegreen}\verb|\ref{<label>}|} і {\color{bluegreen}\verb|\eqref{<label>}|} (остання для посилання на формули), які разом з командою {\color{bluegreen}\verb|\label{<label>}|} генерують перехресні посилання, після клацання на які здійснюється перехід на цитований об'єкт. Тут  {\color{bluegreen}\verb|<label>|} є унікальною міткою (мітки не повинні повторюватися впродовж усього тексту), якою ми маркуємо кожен нумерований об'єкт і яку використовуємо для посилання на нього. Для створення міток використовуються довільні послідовності латинських літер і арабських цифр.
До прикладу, в теоремі Піфагора було створено мітку {\color{bluegreen}\verb|pith|} командою {\color{bluegreen}\verb|\label{pith}|}, тому для посилання на цю теорему ми використовуємо такий текст:

{\color{bluegreen}%
\begin{verbatim}
Розглянемо теорему Піфагора (див. теорему \ref{pith}).
\end{verbatim}%
}\noindent% 
що в тексті має такий вигляд:

\noindent%
Розглянемо теорему Піфагора (див. теорему \ref{pith}). 

Список літератури оформлюється за зразком, наведеним у файлі {\color{bluegreen}\verb|bibliography.tex|}. Кожний елемент цього списку повинен починатися командою {\color{bluegreen}\verb|\bibitem{<cite-label>}|}, де мітка {\color{bluegreen}\verb|<cite-label>|} є унікальною і не повинна збігатися ані з мітками інших елементів списку літератури, ані з мітками інших нумерованих об'єктів впродовж усього тексту. Для посилання на елемент цього списку використовується команда {\color{bluegreen}\verb|\cite{<cite-label>}|}. До прикладу, для посилання на підручник В.М. Радченка, який в списку літератури задається командою
{\color{bluegreen}\verb|\bibitem{VR}|},
використовуємо команду {\color{bluegreen}\verb|\cite{VR}|}, що в результаті дає такий текст: \cite{VR}.
Інші приклади використання перехресних посилань див. у розділах \ref{sec-1} і \ref{sec-2}.

Перед початком роботи в \LaTeX{} дуже бажано ознайомитися с основними принципами створення документів, наприклад, на \url{https://www.overleaf.com/learn} або \url{https://en.wikibooks.org/wiki/LaTeX}.  




\chapter{ТЕОРЕТИЧНЕ ПІДҐРУНТЯ КРИПТОСИСТЕМИ}\label{ch:-1:---}

У цьому розділі викладено основні теоретичні положення, що лежать в основі розробки та аналізу симетричної криптосистеми, запропонованої у цій роботі.


\section{Вступ до основ криптографії}
\label{sec:crypto_intro}

Криптографія пройшла шлях від простих шифрів заміни, що використовувалися ще в античності (наприклад, шифр Цезаря),
до сучасних протоколів захисту інформації (зокрема, TLS, системи наскрізного шифрування).
Основними завданнями криптографії є забезпечення конфіденційності (захист від несанкціонованого доступу),
цілісності (контроль змін) та автентифікації (верифікація ідентичності).

Побудова надійних криптосистем ґрунтується на двох фундаментальних принципах:
\begin{itemize}
    \item \emph{Принцип Керкгоффса}~\cite{Kerckhoffs83}: Безпека криптосистеми не повинна залежати від секретності алгоритму;
    достатньо зберігати в таємниці лише ключ.
    \item \emph{Досконала секретність за Шенноном}~\cite{Shannon49}: Шифр вважається досконалим, якщо шифротекст не містить жодної
    інформації про відкритий текст (наприклад, одноразовий блокнот за умови випадковості ключа).
\end{itemize}

К. Шеннон також увів поняття \emph{перемішування} (confusion) та \emph{розсіювання} (diffusion), які реалізуються у сучасних
блокових шифрах (наприклад, AES) через мережі підстановок і перестановок~\cite{Shannon49}.

Ці високорівневі концепції лежать в основі розробленої криптосистеми.

\subsection{Базові поняття та означення}
\label{subsec:crypto_basics}

Для повного розуміння розробленої системо розглянемо наступні означення~\cite{Mao04,BerczesEtAl14}:

\begin{description}
    \item[Криптосистема:] Трійка алгоритмів \((\mathsf{KeyGen},\mathsf{Encrypt},\mathsf{Decrypt})\), що працюють за поліноміальний час відносно параметра безпеки \(n\).
    \(\mathsf{KeyGen}\) генерує ключі; \(\mathsf{Encrypt}(K,P)\) перетворює відкритий текст \(P\) у шифротекст \(C\); \(\mathsf{Decrypt}(K,C)\) відновлює \(P\).
    \item[Відкритий текст (\(P\)):] Початкове повідомлення.
    \item[Шифротекст (\(C\)):] Закодований результат, що приховує \(P\).
    \item[Ключ (\(K\)):] Секретний параметр, що визначає відображення шифрування та розшифрування.
    \item[Шифрування:] Операція \(C \leftarrow \mathsf{Encrypt}(K,P)\).
    \item[Розшифрування:] Операція \(P \leftarrow \mathsf{Decrypt}(K,C)\), причому \\ \(\mathsf{Decrypt}(K, \mathsf{Encrypt}(K,P)) = P\).
    \item[Криптоаналіз:] Методи відновлення \(K\) або \(P\) з \(C\) без авторизованого доступу.
\end{description}

Для кращого розуміння вище наведених визначень розглянемо наступний приклад.

\begin{example}[Шифр Вернама / Одноразовий блокнот]
    \label{ex:otp_vernam}
    Цей симетричний шифр, також відомий як одноразовий блокнот (One-Time Pad, OTP), був запропонований Гільбертом Вернамом.
    Нехай відкритий текст \(P\), шифротекст \(C\) та ключ \(K\) є бінарними послідовностями однакової довжини \(n\), тобто \(P, C, K \in \{0, 1\}^n\).
    Операції шифрування та розшифрування виконуються побітово за допомогою операції XOR (\(\oplus\), що еквівалентно додаванню за модулем 2):
    \begin{itemize}
        \item Шифрування: \(C = P \oplus K\) (тобто \(C_i = P_i \oplus K_i\) для \(i=1, \dots, n\)).
        \item Розшифрування: \(P = C \oplus K\) (оскільки \(C \oplus K = (P \oplus K) \oplus K = P \oplus (K \oplus K) = P \oplus 0 = P\)).
    \end{itemize}
    Шифр Вернама забезпечує досконалу секретність за Шенноном, тобто є теоретично невразливим, за умови суворого дотримання таких вимог до ключа \(K\):
    \begin{enumerate}
        \item Ключ \(K\) має бути абсолютно випадковою послідовністю.
        \item Довжина ключа \(|K|\) повинна бути не меншою за довжину відкритого тексту \(|P|\).
        \item Ключ \(K\) повинен використовуватися для шифрування лише одного повідомлення (одноразове використання).
    \end{enumerate}
    Порушення будь-якої з цих умов робить шифр вразливим.
    Основною практичною складністю є генерація та безпечна передача довгих, випадкових одноразових ключів~\cite{Shannon49}.
\end{example}

Криптографічні системи можемо поділити на \emph{симетричні криптосистеми}, де один і той самий секретний ключ \(K\) використовується для шифрування і розшифрування.
До прикладів можемо віднести: шифр Вернама (який наведено як приклад вище), AES, ChaCha20.
Та на \emph{Асиметричні криптосистеми} (з відкритим ключем), де для шифрування використовується відкритий ключ \(K_{\text{pub}}\), а для розшифрування — приватний ключ \(K_{\text{priv}}\).
До прикладів можемо віднести RSA, ElGamal encryption.

Запропонована система належить до симетричних і ґрунтується на сюр'єктивних відображеннях скінченних кілець та їх ізоморфізмів з використанням систем лінійних рівнянь над кільцями лишків.

\subsection{Мотивація та підхід створення криптосистеми}
\label{subsec:motivation}

Метою цієї роботи є розробка криптосистеми, що базується на об'єктах порівняно невеликих розмірів і забезпечує достатній рівень стійкості до злому.
Можна виділити наступні ключові характеристики системи:

\begin{enumerate}
    \item \emph{Обчислювальна ефективність:} Використання малого модуля \(k\) зменшує вартість арифметичних операцій.
    \item \emph{Простота реалізації:} Скінченні кільця \(\mathbb{Z}_k\) легко реалізуються на пристроях з обмеженими ресурсами (наприклад, смарт-карти, IoT).
    \item \emph{Комбінаторна стійкість:} Кількість сюр'єктивних гомоморфізмів \(\mathbb{Z}_k \to \mathbb{Z}_\ell\) та ізоморфізмів кілець швидко зростає зі зростанням кількості дільників \(k\), що ускладнює криптоаналіз.
    \item \emph{Стійкість до статистичних атак:} Вбудовування відкритого тексту у системи лінійних рівнянь над \(\mathbb{Z}_k\) ускладнює застосування частотного аналізу та подібних методів.
\end{enumerate}

Запропонована схема розвиває ідеї симетричного протоколу обміну~\cite{KryvyiEtAl22}, доповнюючи їх кільцевими перетвореннями та афінними відображеннями систем лінійних рівнянь.


\section{Основи теорії кілець}
\label{sec:ring_theory}

Кільця становлять алгебраїчну основу для побудови криптосистеми.
У цьому підрозділі наведено їхні означення, властивості та приклади, що мають значення для подальшого викладу.

\subsection{Означення та властивості кілець}
\label{subsec:ring_definition}

\begin{definition}
    \label{def:ring}
    \emph{Кільце} \((R,+,\cdot)\) — це множина \(R\) з двома бінарними операціями, для яких виконуються такі умови:
    \begin{enumerate}
        \item \((R,+)\) — абелева група з нульовим елементом \(0\) та протилежним елементом \(-a\) для кожного \(a\in R\).
        \item \((R,\cdot)\) — моноїд з одиницею \(1\): множення асоціативне (\(a(bc)=(ab)c\)), існує \(1\in R\) таке, що \(1\cdot a = a\cdot 1 = a\) для всіх \(a\in R\).
        \item \emph{Розподільність:} для всіх \(a,b,c\in R\)
        \[
            a\cdot(b + c) = ab + ac, \quad (b + c)\cdot a = ba + ca.
        \]
    \end{enumerate}
    Якщо множення комутативне (\(ab=ba\)), кільце називають \emph{комутативним}.
    Якщо \(R\) скінченне, комутативне та має одиницю \(1\neq0\), таке кільце називають \emph{скінченним комутативним кільцем з одиницею}.
\end{definition}

Основні поняття, пов'язані з кільцями:
\begin{description}
    \item[Дільник нуля:] Елемент \(a\neq0\), для якого існує \(b\neq0\) такий, що \(ab=0\).
    \item[Дільник одиниці (оборотний елемент):] Елемент \(u\in R\), для якого існує обернений \(u^{-1}\) такий, що \(uu^{-1}=u^{-1}u=1\).
    Множина всіх дільників одиниці утворює групу відносно множення, позначається \(R^\times\).
    \item[Характеристика:] Найменше натуральне \(n>0\), для якого \(n\cdot1 = 0\) (тобто \(1+\dots+1=0\), \(n\) разів).
    Якщо такого \(n\) не існує, характеристика дорівнює нулю.
\end{description}

\subsection{Кільця лишків за модулем \(k\) (\(\mathbb{Z}_k\))}
\label{subsec:residue_rings}

Нехай \(k\ge2\) — ціле число.
Кільце лишків за модулем \(k\) — це множина класів конгруентності за модулем \(k\):
\[
    \mathbb{Z}_k = \{\bar{0},\bar{1},\dots,\overline{k-1}\}, \quad \text{де } \bar{a} = a + k\mathbb{Z} = \{x\in\mathbb{Z} : x \equiv a \pmod{k}\}.
\]
Операції визначаються так:
\[
    \bar{a} + \bar{b} = \overline{a + b}, \quad \bar{a}\cdot\bar{b} = \overline{a\cdot b},
\]
тобто додавання та множення виконуються за модулем \(k\).
\((\mathbb{Z}_k, +, \cdot)\) — скінченне комутативне кільце з одиницею \(\bar{1}\).

\begin{example}[Операції в \(\mathbb{Z}_8\)]
    \label{ex:z8_ops}
    У кільці \(\mathbb{Z}_8\):
    \begin{gather*}
        \bar{3} + \bar{6} = \overline{3 + 6} = \overline{9} \equiv \bar{1} \pmod{8}.\\
        \bar{3}\cdot\bar{6} = \overline{3 \cdot 6} = \overline{18} \equiv \bar{2} \pmod{8}.\\
    \end{gather*}
    Дільники одиниці: \(\mathbb{Z}_8^\times = \{\bar{a} \in \mathbb{Z}_8 : \gcd(a,8)=1\} = \{\bar{1},\bar{3},\bar{5},\bar{7}\}\).
    Дільники нуля: \(\{\bar{a} \in \mathbb{Z}_8 \setminus \{\bar{0}\} : \gcd(a,8)>1\} = \{\bar{2},\bar{4},\bar{6}\}\).
    Наприклад, \(\bar{2}\cdot\bar{4} = \bar{8} \equiv \bar{0} \pmod{8}\).
\end{example}

\begin{example}[Ізоморфізм кілець]
    \label{ex:crt_iso}
    За китайською теоремою про лишки, якщо \(k={k_1k_2}\) і \(\gcd(k_1,k_2)=1\), то кільця \(\mathbb{Z}_k\) та \(\mathbb{Z}_{k_1} \times \mathbb{Z}_{k_2}\) є ізоморфними: \(\mathbb{Z}_k \cong \mathbb{Z}_{k_1} \times \mathbb{Z}_{k_2}\).
    Наприклад, \(\mathbb{Z}_6 \cong \mathbb{Z}_2 \times \mathbb{Z}_3\).
    Ізоморфізм \(\psi: \mathbb{Z}_6 \to \mathbb{Z}_2 \times \mathbb{Z}_3\) задається як \(\psi(\bar{a}) = (\overline{a \bmod 2}, \overline{a \bmod 3})\).
    Наприклад, \(\psi(\bar{5}) = (\overline{5 \bmod 2}, \overline{5 \bmod 3}) = (\bar{1}, \bar{2})\).
\end{example}

Позначимо через \(G_k\) будь-яке кільце, ізоморфне \(\mathbb{Z}_k\), з фіксованим ізоморфізмом \(\varphi: G_k \to \mathbb{Z}_k\).

\subsection{Мультиплікативна група дільників одиниці}
\label{subsec:ring_units_group}

Множина дільників одиниці \(\mathbb{Z}_k^\times\) утворює групу відносно множення за модулем \(k\):
\[
    \mathbb{Z}_k^\times = \{\bar{a}\in\mathbb{Z}_k : \gcd(a,k)=1\}.
\]
Ця група є абелевою.
Її порядок дорівнює \(\varphi(k)\), де \(\varphi\) — функція Ойлера.

\begin{example}[Група \(\mathbb{Z}_{10}^\times\)]
    \label{ex:z10_units}
    У \(\mathbb{Z}_{10}\), дільники одиниці: \(\mathbb{Z}_{10}^\times = \{\bar{1},\bar{3},\bar{7},\bar{9}\}\).
    Порядок групи \(\varphi(10) = 4\).
    Операція — множення за модулем 10.
    Наприклад, \(\bar{3}\cdot\bar{7} = \overline{21} \equiv \bar{1} \pmod{10}\), отже \(\bar{7} = \bar{3}^{-1}\).
    Група \(\mathbb{Z}_{10}^\times\) є циклічною, оскільки \(\bar{3}\) є генератором групи.
\end{example}

\begin{theorem}[Гаус~\cite{Shoup08}]
    \label{thm:cyclic_units}
    Група \(\mathbb{Z}_k^\times\) є циклічною тоді і тільки тоді, коли \(k=1,2,4,p^m\) або \(k=2p^m\), де \(p\) — непарне просте число, \(m\geq1\).
\end{theorem}

\subsection{Відображення між кільцями}
\label{subsec:ring_mappings}

Розглядаються такі типи відображень між кільцями:

\begin{description}
    \item[Гомоморфізм \(\phi: R\to S\):] Відображення, що зберігає операції: \(\phi(a+b)=\phi(a)+\phi(b)\) та \(\phi(a\cdot b)=\phi(a)\cdot\phi(b)\) для всіх \(a,b\in R\).
    \item[Ізоморфізм \(\varphi: R\to S\):] Бієктивний гомоморфізм.
    Якщо існує ізоморфізм, кільця \(R\) і \(S\) структурно еквівалентні (\(R\cong S\)).
    \item[Сюр'єкція (епіморфізм) \(\psi: R\to T\):] Гомоморфізм, образ якого збігається з усім \(T\) (\(\mathrm{Im}\,\psi = T\)).
    \item[Бієкція \(\psi_1: R\to R'\):] Взаємно однозначне відображення між множинами \(R\) та \(R'\), яке не обов'язково зберігає операції кілець.
\end{description}

\subsection{Ідеали та фактор-кільця}
\label{subsec:factor_rings}

Нехай \(R\) — кільце. Для побудови фактор-кільця необхідно ввести поняття ідеалу.

\textbf{Ідеал.} Підмножина \(I \subseteq R\) називається \emph{ідеалом} кільця \(R\), якщо виконуються такі умови:
\begin{itemize}
    \item \(I\) є підгрупою відносно додавання: для будь-яких \(a, b \in I\) маємо \(a-b \in I\);
    \item \(I\) замкнена відносно множення на довільний елемент кільця: для будь-яких \(r \in R\), \(a \in I\) маємо \(ra \in I\) і \(ar \in I\).
\end{itemize}
Ідеал можна розглядати як "узагальнення" поняття кратних у кільці цілих чисел: наприклад, множина всіх чисел, кратних \(n\), є ідеалом у \(\mathbb{Z}\).

\textbf{Фактор-кільце.} Нехай \(I\) — ідеал у кільці \(R\). \emph{Фактор-кільце} \(R/I\) — це множина всіх класів суміжності за ідеалом \(I\), тобто множина підмножин вигляду:
\[
    a + I = \{a + r : r \in I\}, \quad \text{де } a \in R.
\]
Кожен елемент \(R\) належить рівно одному такому класу.

Операції додавання та множення на \(R/I\) визначаються так:
\begin{gather*}
    (a + I) + (b + I) = (a + b) + I, \\
    (a + I) \cdot (b + I) = (ab) + I.
\end{gather*}
Ці операції коректно визначені, тобто не залежать від вибору представників класів.

\textbf{Властивість.} Фактор-кільце \(R/I\) саме є кільцем, а природне відображення \(\pi: R \to R/I\), \(\pi(a) = a + I\), є гомоморфізмом кілець з ядром \(I\).

\textbf{Теорема про ізоморфізм.} Якщо \(\psi: R \to T\) — сюр'єктивний гомоморфізм кілець, то фактор-кільце \(R/\ker\psi\) ізоморфне образу \(\psi\), тобто \(R/\ker\psi \cong \mathrm{Im}(\psi)\). Зокрема, якщо \(\psi\) сюр'єктивне, то \(R/\ker\psi \cong T\).

Таким чином, фактор-кільце дозволяє \texttt{"}розділити\texttt{"} кільце на класи за ідеалом і отримати нову алгебраїчну структуру, яка часто має простішу або більш зручну для застосування будову.

\section{Системи лінійних рівнянь над кільцями}
\label{sec:sle_theory}

У цьому підрозділі розглядаються системи лінійних рівнянь (СЛР) над скінченними комутативними кільцями.

\subsection{Означення та матричний запис}
\label{subsec:sle_definition}

Нехай \(R = \mathbb{Z}_m\) — скінченне комутативне кільце з одиницею.
\emph{Система з \(n\) лінійних рівнянь з \(n\) невідомими} над \(R\) має вигляд:
\[
    \begin{cases}
        a_{11}x_1 + a_{12}x_2 + \cdots + a_{1n}x_n \;\equiv\; b_1 \pmod{m},\\
        a_{21}x_1 + a_{22}x_2 + \cdots + a_{2n}x_n \;\equiv\; b_2 \pmod{m},\\
        \quad\vdots\\
        a_{n1}x_1 + a_{n2}x_2 + \cdots + a_{nn}x_n \;\equiv\; b_n \pmod{m},
    \end{cases}
\]
де коефіцієнти \(a_{ij} \in \mathbb{Z}_m\), вільні члени \(b_i \in \mathbb{Z}_m\), а невідомі \(x_j\) шукаються в \(\mathbb{Z}_m\).

У матричному записі ця система має вигляд:
\[
    A\,x \;\equiv\; b \pmod{m},
\]
де \(A = (a_{ij})\) — матриця розміру \(n\times n\) з елементами з \(\mathbb{Z}_m\), \(x = (x_j)\) — стовпчиковий вектор невідомих розміру \(n\times 1\), \(b = (b_i)\) — стовпчиковий вектор вільних членів розміру \(n\times 1\).

\subsection{Лінійні діофантові рівняння та конгруенції}
\label{subsec:diophantine}

\emph{Лінійне діофантове рівняння} — це рівняння вигляду \(a_1 x_1 + \dots + a_k x_k = c\) з цілими коефіцієнтами \(a_i, c \in \mathbb{Z}\), де шукаються цілі розв'язки \((x_1,\dots,x_k)\)~\cite{KameswariEtAl21}.
Система лінійних конгруенцій за модулем \(m\), як визначено вище, тісно пов'язана з діофантовими рівняннями.
Кожна конгруенція \( \sum_j a_{ij} x_j \equiv b_i \pmod{m} \) еквівалентна лінійному діофантовому рівнянню \( \sum_j a_{ij} x_j - m y_i = b_i \) для деякого цілого числа \(y_i\).

Окрема лінійна конгруенція \(a x \equiv c \pmod{m}\) має розв'язок для \(x \in \mathbb{Z}_m\) тоді й лише тоді, коли \(d = \gcd(a,m)\) ділить \(c\).
Якщо розв'язок існує, то їх рівно \(d\) за модулем \(m\)~\cite{Kryvyi21}.

Розв'язання систем лінійних рівнянь над \(\mathbb{Z}_m\) у цій роботі ґрунтується на алгоритмах, описаних у~\cite{Kryvyi21}.

\subsection{Перетворення систем}
\label{subsec:sle_transformations}

Для обфускації системи \(A\,x \equiv b \pmod m\) можна застосовувати афінне перетворення змінних, яке задається оберненою матрицею \(B \in \mathrm{GL}_n(\mathbb{Z}_m)\) та вектором зсуву \(a \in \mathbb{Z}_m^n\):
\[
    x = B y + a.
\]
Підставляючи це у вихідну систему \(Ax \equiv b \pmod m\), отримуємо:
\[
    (A B) y \equiv b - A a \pmod{m}
\]
Позначимо \(A' = A B\) та \(b' = b - A a\).
Отримана перетворена система має вигляд \(A' y \equiv b' \pmod m\).

Інший тип перетворення — домноження обох частин системи зліва на обернену матрицю \(C \in \mathrm{GL}_n(\mathbb{Z}_m)\):
\[
    (C A) x \equiv C b \pmod m.
\]

Комбінування цих перетворень та їх композиція з ізоморфізмами чи гомоморфізмами кілець можуть змінювати подання системи.


%\chapter{Породжені класи множин}
\label{sec-2}

\section{Означення і елементарні властивості породжених класів множин}

Спочатку розглянемо означення породжених класів множин.

\begin{definition}
\label{def-1-7}
Нехай $\HH\subset2^X$ є непорожнім класом множин.
Клас
\begin{equation}
\label{1.4}
k(\HH)=\bigcap_{\substack{\K_\alpha\supset \HH\\ \K_\alpha\text{ є кільцем}}} \K_\alpha
\end{equation}
називається \emph{кільцем, породженим класом $\HH$}; клас
\begin{equation}
\label{1.4-1}
a(\HH)=\bigcap_{\substack{\mathcal A_\alpha\supset \HH\\ \mathcal A_\alpha\text{ є алгеброю}}} \mathcal A_\alpha
\end{equation}
називається \emph{алгеброю, породженою класом $\HH$}; клас
\begin{equation}
\label{1.4-2}
\sigma k(\HH)=\bigcap_{\substack{\K_\alpha\supset \HH\\ \K_\alpha\text{ є $\sigma$-кільцем}}} \K_\alpha
\end{equation}
називається \emph{$\sigma$-кільцем, породженим класом $\HH$}; клас
\begin{equation}
\label{1.4-3}
\sigma a(\HH)=\bigcap_{\substack{\mathcal A_\alpha\supset \HH\\ \mathcal A_\alpha\text{ є $\sigma$-алгеброю}}} \mathcal A_\alpha
\end{equation}
називається \emph{$\sigma$-алгеброю, породженою класом $\HH$}; клас
\begin{equation}
\label{1.4-4}
m(\HH)=\bigcap_{\substack{\mathcal M_\alpha\supset \HH\\ \mathcal M_\alpha\text{ є монотонним}\\ \text{\quad класом}}} \mathcal M_\alpha
\end{equation}
називається \emph{монотонним класом, породженим класом $\HH$}.
\end{definition}

Доведемо, що породжені класи множин зберігають структуру класів, які їх утворюють.
\begin{statement}
\label{prop-gen-str}
Справедливі наступні твердження:
\begin{enumerate}
\item \label{prop-gen-st-1}
$k(\HH)$ є кільцем;
\item \label{prop-gen-str-2}
$a(\HH)$ є алгеброю;
\item \label{prop-gen-str-3}
$\sigma k(\HH)$ є $\sigma$-кільцем;
\item \label{prop-gen-str-4}
$\sigma a(\HH)$ є $\sigma$-алгеброю;
\item \label{prop-gen-str-5}
$m(\HH)$ є монотонним класом.
\end{enumerate}
\end{statement}

\begin{proof}
Усі наведені твердження доводяться цілком аналогічно. Тому ми доводимо лише твердження  \ref{prop-gen-st-1}. Скористаємося означенням кільця (див. означення \ref{def-1-1}).  Клас  $k(\HH)$ містить  клас $\HH$, який є непорожнім за означенням. Отже, $k(\HH)\neq\emptyset$.  Нехай $A\in k(\HH)$ і $B\in k(\HH)$. Тоді  $A\in\K_\alpha$ і $B\in\K_\alpha$, де $\K_\alpha$ є довільним кільцем, що містить клас $\HH$. За означенням кільця (див. означення \ref{def-1-1}) маємо $A\cup B\in\K_\alpha$ і $A\setminus B\in\K_\alpha$. Оскільки $\K_\alpha$ є довільним кільцем, що містить клас $\HH$, за означенням кільця, породженого цим класом, маємо $A\cup B\in k(\HH)$ і $A\setminus B\in k(\HH)$.
\end{proof}

Доведене твердження дає підставу називати $k(\HH)$ \emph{мінімальним кільцем, що містить клас $\HH$}, $a(\HH)$ \emph{мінімальною алгеброю, що містить клас $\HH$}, $\sigma k(\HH)$ \emph{мінімальним $\sigma$-кільцем, що містить клас $\HH$}, $\sigma a(\HH)$ \emph{мінімальною $\sigma$-алгеброю, що містить клас $\HH$}, $m(\HH)$ \emph{мінімальним монотонним класом, що містить клас $\HH$}.

\begin{example}
	\label{ex-1-9}
	Нехай $X$ є довільною нескінченною множиною, $\HH$ є класом усіх одноточкових підмножин $X$. Тоді $k(\HH)$ складається з усіх скінченних підмножин $X$.
\end{example}

Наступна \emph{теорема про структуру кільця, породженого півкільцем}, узагальнює цей приклад.

\begin{theorem}
\label{th-1-3}
Нехай $\PP\subset2^X$ є півкільцем. Тоді
\begin{equation}
\label{1.5}
k(\PP)=\left\{\bigcup_{k=1}^n A_k \midd n\in\N \land \PP\supset\{A_k\}_{k=1}^n\ \text{є неперетинними} \right\}.
\end{equation}
\end{theorem}

\begin{proof}
Клас множин у правій частині \eqref{1.5} позначимо через $\mathcal L$. Доведемо, що  $\mathcal L\subset k(\PP)$ і $\mathcal L\supset k(\PP)$.

Спочатку доведемо, що $\mathcal L\subset k(\PP)$. Нехай $A\in\mathcal L$.  Тоді існує набір неперетинних множин $\{A_k\}_{k=1}^n\subset\PP$  такий, що $\displaystyle A=\bigcup_{k=1}^n A_k $. Оскільки $\PP\subset k(\PP)$  і $k(\PP)$ є кільцем, то за умовою \ref{def-1-1-i} означення кільця (див. означення \ref{def-1-1}) маємо  $\displaystyle A=\bigcup_{k=1}^n A_k\in k(\PP)$. Отже,  $\mathcal L\subset k(\PP)$.

Тепер доведемо, що $\mathcal L\supset k(\PP)$. Оскільки $\PP\subset\mathcal L$,   то досить довести, що $\mathcal L$ є кільцем і скористатися означенням кільця, породженого заданим класом множин (див. означення \ref{def-1-7}).

Нехай $A\in\mathcal L$  і $B\in\mathcal L$ . Тоді існує набір неперетинних множин $\{A_k\}_{k=1}^n\subset\PP$  і набір неперетинних множин $\{B_i\}_{i=1}^j\subset\PP$ такі, що
$$
 A=\bigcup_{k=1}^n A_k \quad\text{і}\quad  B=\bigcup_{i=1}^j B_i.
$$

Розіб'ємо подальше доведення на чотири кроки.
\begin{enumerate}[label={\slshape Крок } {\upshape\arabic*.}, ref={\upshape \arabic*}, leftmargin=0em, labelwidth=53pt, itemindent=70.37pt, labelsep=0pt, align=left]
\item \label{th-1-3-1}
Якщо $A\cap B=\emptyset$, то $A_k\cap B_i=\emptyset$, $k=\overline{1,n}$, $i=\overline{1,j}$,\the\itemindent
$$
A\cup B=\left(\bigcup_{k=1}^n A_k\right) \cup \left(\bigcup_{i=1}^j B_i\right) \in\mathcal L
$$
як об'єднання неперетинних елементів $\PP$.
\item \label{th-1-3-2}
Оскільки $\PP$ є півкільцем, $A_k\cap B_i\in \PP$, $k=\overline{1,n}$, $i=\overline{1,j}$. Крім того, множини $\{A_k\cap B_i\}_{\substack{k=\overline{1,n}\\i=\overline{1,j} }}$   є неперетинними.  Тому
$$
A\cap B=\left(\bigcup_{k=1}^n A_k\right) \cap \left(\bigcup_{i=1}^j B_i\right) =\bigcup_{k=1}^n\bigcup_{i=1}^j (A_k\cap B_i)  \in\mathcal L
$$
як об'єднання неперетинних елементів $\PP$  (див. рис. \ref{fig-1-5}).
\begin{figure}[!h]
	\centering
	\includegraphics{fig-1-5}
	\caption{Подання множин $A$ і $B$.}
	\label{fig-1-5}
\end{figure}
\item \label{th-1-3-3}
Маємо (див. рис. \ref{fig-1-5}) 
\begin{align}
\label{th-1-3.1}
A\setminus B&= \bigcup_{k=1}^n (A_k\setminus B),
\\
\label{th-1-3.2} 
A_k\setminus B&= A_k\setminus \left(\bigcup_{i=1}^j B_i\right)
= \bigcap_{i=1}^j (A_k\setminus B_i),\quad k=\overline{1,n}.
\end{align}
Для будь-яких $k=\overline{1,n}$ і $i=\overline{1,j}$ маємо $A_k\in \PP$ і $B_i\in\PP$, тому
$$
A_k\setminus B_i=\bigcup_{r=1}^{s} C_r^{ki},
$$
де $\PP\supset\{C_r^{ki}\}_{r=1}^s$ є неперетинними. Отже, $A_k\setminus B_i\in\mathcal L$, $k=\overline{1,n}$, $i=\overline{1,j}$. Ураховуючи \eqref{th-1-3.2}, згідно з кроком \ref{th-1-3-2} маємо 
$A_k\setminus B\in\mathcal L$, $k=\overline{1,n}$.  Оскільки $\{A_k\setminus B\}_{k=1}^n$  є неперетинними, беручи до уваги крок \ref{th-1-3-1} і \eqref{th-1-3.1}, одержуємо $A\setminus B\in\mathcal L$.
\item \label{th-1-3-4}
Маємо
$$
A\cup B=(A\setminus B)\cup(A\cap B)\cup(B\setminus A),
$$
де $A\setminus B$, $A\cap B$, $B\setminus A$ є неперетинними. Згідно з кроком \ref{th-1-3-2}  $A\cap B\in\mathcal L$,  а згідно з кроком \ref{th-1-3-3} $A\setminus B\in\mathcal L$ і $B\setminus A\in\mathcal L$. Тому, беручи до уваги крок \ref{th-1-3-1}, одержуємо $A\cup B\in\mathcal L$. 
\begin{figure}[!h]
	\centering
	\includegraphics{fig-1-6}
	\caption{Подання об'єднання множин $A$ і $B$.}
	\label{fig-1-6}
\end{figure}
\end{enumerate}

Таким чином, $\mathcal L$ є кільцем.    Оскільки $\PP\subset\mathcal L$ за побудовою,    скориставшись означенням кільця, породженого заданим класом множин (див. означення \ref{def-1-7}), одержуємо $\mathcal L\supset k(\PP)$.  Крім того, ми вже довели, що $\mathcal L\subset k(\PP)$. Отже, $\mathcal L= k(\PP)$, тобто \eqref{1.5}  виконано. Теорему доведено.
\end{proof}

\begin{example}
\label{ex-1-10}
Згідно з цією теоремою для півкільця $\PP_1$ з прикладу  \ref{ex-1-4} маємо
$$
k(\PP_1)=\left\{\bigcup_{k=1}^n (a_k,b_k] \midd n\in\N \land \R\supset\{(a_k,b_k]\}_{k=1}^n \ \text{є неперетинними} \right\}.
$$
\end{example}

\begin{example}
\label{ex-1-10-a}
Згідно з цією теоремою  для півкільця $\PP_d$ з прикладу  \ref{ex-1-5} маємо
\begin{align*}
k(\PP_d)&=\left\{\bigcup_{k=1}^n \left(\bigtimes_{m=1}^d\left(a_m^k,b_m^k\right]\right) \midd
\right.
\\
&\hspace{25mm}\left.
 n\in\N\ \land\ \R^d\supset\left\{ \bigtimes_{m=1}^d\left(a_m^k,b_m^k\right]\right\}_{k=1}^n \ \text{є неперетинними} \right\}.
\end{align*}
\end{example}

\section{Дві теореми про породжені класи}

\begin{definition}
	\label{intersec}
	Нехай $B\subset X$ і $\HH\subset 2^X$. Позначимо
	$$
	\HH\cap B=\{a\cap B \mid A\in\HH\}.
	$$
\end{definition}

Розглянемо теорему про перетин $\sigma$-кільця.

\begin{theorem}
	\label{th-1-4}
	Нехай $B\subset X$ і $\HH\subset 2^X$. Тоді
	$$
	\sigma k(\HH\cap B)=\sigma k(\HH)\cap B.
	$$
\end{theorem}

\begin{proof}
	Доведення цієї рівності розіб'ємо на дві частини: $\sigma k(\HH\cap B)\subset\sigma k(\HH)\cap B$ і $\sigma k(\HH\cap B)\supset\sigma k(\HH)\cap B$.
	
	Нехай $\{A_n\}_{n=1}^\infty\subset \sigma k(\HH)$. Маємо  $\displaystyle \bigcup_{n=1}^\infty A_n\in \sigma k(\HH)$, $A_1\setminus A_1\in \sigma k(\HH)$, $\{A_n\cap B\}_{n=1}^\infty\subset \sigma k(\HH)\cap B$ і (див. рис. \ref{fig-1-7})
	\begin{align*}
	\bigcup_{n=1}^\infty (A_n\cap B)& = \left(\bigcup_{n=1}^\infty A_n\right)\cap B \in \sigma k(\HH)\cap B,
	\\
	(A_1\cap B)\setminus (A_2\cap B)&= (A_1\setminus A_2) \cap B\in \sigma k(\HH)\cap B.
	\end{align*}
	\begin{figure}[!h]
		\centering
		\includegraphics{fig-1-7}
		\caption{Подання об'єднання множин $A$ і $B$.}
		\label{fig-1-7}
	\end{figure}
	Тому $\sigma k(\HH)\cap B$ є $\sigma$-кільцем. Оскільки $\HH\subset \sigma k(\HH)$, маємо $\HH\cap B \subset \sigma k(\HH)\cap B$. Отже, $\sigma k(\HH\cap B)\subset\sigma k(\HH)\cap B$ тому, що $\sigma k(\HH)\cap B$ є $\sigma$-кільцем.
	
	Далі розглянемо клас множин
	\begin{equation}
	\label{1.7}
	\mathcal L= \big\{A\cup(C\setminus B) \mid A\in \sigma k(\HH\cap B) \land C\in \sigma k(\HH)\big\}.
	\end{equation}
	
	Перевіримо, що $\mathcal L$ є $\sigma$-кільцем. Нехай $\{A_n\}_{n=1}^\infty\subset \sigma k(\HH\cap B)$, $C\in \sigma k(\HH)$. Тоді $\displaystyle \bigcup_{n=1}^\infty A_n\in \sigma k(\HH\cap B)$, $A_1\setminus A_2\in \sigma k(\HH\cap B)$, $\{A_n\cup(C\setminus B)\}_{n=1}^\infty\subset \mathcal L$ і
	\begin{align*}
	\bigcup_{n=1}^\infty (A_n\cup(C\setminus B))& = \left(\bigcup_{n=1}^\infty A_n\right)\cup(C\setminus B) \in \mathcal L,
	\\
	(A_1\cup(C\setminus B))\setminus (A_2\cup(C\setminus B))&= (A_1\setminus A_2) \cup(C\setminus B)\in \mathcal L
	\end{align*}
	(див. рис. \ref{fig-1-7}, де в якості множини $B$  розглядається $C\setminus B$).
	Тому $\mathcal L$ є $\sigma$-кільцем. Для $D\in\HH$ маємо $D\in \sigma k(\HH)$ і
	$$
	D=(D\cap B)\cup (D\setminus B).
	$$
	Оскільки $D\cap B\subset \HH\cap B\subset \sigma k(\HH\cap B)$, маємо $D\in\mathcal L$. Отже, $\HH\subset \mathcal L$. З того, що $\mathcal L$ є $\sigma$-кільцем випливає $\sigma k(\HH)\subset \mathcal L$. Тому $\sigma k(\HH)\cap B\subset \mathcal L\cap B=\sigma k(\HH\cap B)$ в останній рівності враховано співвідношення \eqref{1.7}. Теорему доведено. 
\end{proof}

Розглянемо теорему про рівність монотонного класу і $\sigma$-кільця, породженого кільцем.

\begin{theorem}
	\label{th-1-5}
	Нехай $\K\subset 2^X$ є кільцем. Тоді $m(\K)=\sigma k(\K)$.
\end{theorem}

\begin{proof}  
	Оскільки клас $\sigma k(\K)$ є $\sigma$-кільцем, цей клас є замкненим відносно взяття нескінченного об'єднання (за означенням \ref{def-1-4}) і нескінченного перетину (за теоремою \ref{st-pr-sring}) тому він є монотонним класом. Ураховуючи те, що  $\K\subset \sigma k(\K)$, одержуємо $m(\K)\subset \sigma k(\K)$.
	
	Доведемо тепер протилежне включення. Спочатку доведемо, що $m(\K)$ є кільцем.
	
	Зафіксуємо довільну множину $D\in m(\K)$ і розглянемо клас множин
	$$
	\LL(D)=\big\{C\in X\mid D\cup C\in m(\K) \land D\setminus C\in m(\K) \land C\setminus D\in m(\K) \big\}.
	$$
	Перевіримо, що $\LL(D)$ є монотонним класом. Нехай $\{C_n\}_{n=1}^\infty\uparrow\subset \LL(D)$ і
	$$
	C=\lim_{n\to\infty} C_n=\bigcup_{n=1}^\infty C_n.
	$$ 
	Маємо $\{D\cup C_n\}_{n=1}^\infty\uparrow\subset m(\K)$, $\{D\setminus C_n\}_{n=1}^\infty\downarrow\subset m(\K)$, $\{ C_n\setminus D\}_{n=1}^\infty\uparrow\subset m(\K)$. Оскільки $m(\K)$ є монотонним класом, маємо
	\begin{align*}
	D\cup C&=D\cup \left(\bigcup_{n=1}^\infty C_n\right) = \bigcup_{n=1}^\infty (D\cup C_n)\in m(\K),
	\\
	D\setminus C&=D\setminus \left(\bigcup_{n=1}^\infty C_n \right)= \bigcap_{n=1}^\infty (D\setminus C_n)\in m(\K),
	\\
	C\setminus D&=\left(\bigcup_{n=1}^\infty C_n\right) \setminus D = \bigcup_{n=1}^\infty (C_n\setminus D)\in m(\K),
	\end{align*}
	тобто $C\in\LL(D)$. Цілком аналогічно перевіряємо, що для $\{C_n\}_{n=1}^\infty\downarrow\subset \LL(D)$ і
	$$
	C=\lim_{n\to\infty} C_n=\bigcap_{n=1}^\infty C_n\in\LL(D).
	$$ 
	Отже, $\LL(D)$ є монотонним класом.
	
	Якщо $E\in\K$ і $F\in\K$, то
	$$
	E\cup F\in\K\subset m(\K),\quad E\setminus F\in\K\subset m(\K),\quad F\setminus E\in\K\subset m(\K)
	$$
	тому, що $\K$ є кільцем. Отже, $F\in\LL(E)$ і $\K\subset\LL(E)$ оскільки $F\in\K$ вибрано довільним. З того, що $\LL(E)$ є монотонним класом випливає $m(\K)\subset\LL(E)$ для $E\in\K$.
	
	Отже, для будь-якої множини $B\in m(\K)$ маємо $B\in\LL(E)$, тому 
	$$
	E\cup B\in m(\K),\quad E\setminus B\in m(\K),\quad B\setminus E\in m(\K).
	$$
	Це означає, що $E\in\LL(B)$. Отже, $\K\subset\LL(B)$ оскільки $E\in\K$ вибрано довільним. З того, що $\LL(B)$ є монотонним класом випливає $m(\K)\subset\LL(B)$ тепер вже для $B\in m(\K)$.
	
	Нехай $A\in m(\K)$ і $B\in m(\K)$. Тоді $A\in\LL(B)$ і
	$$
	A\cup B\in m(\K),\quad A\setminus B\in m(\K),\quad B\setminus A\in m(\K).
	$$
	Отже, $m(\K)$ є кільцем.
	Оскільки $m(\K)$ є монотонним класом і кільцем одночасно, то за теоремою \ref{th-1-2} про зв'язок між кільцем і монотонним класом $m(\K)$ є $\sigma$-кільцем. З того, що $\K\subset m(\K)$ випливає $\sigma k(\K)\subset m(\K)$. Оскільки протилежне включення також є вірним, то $\sigma k(\K)= m(\K)$. Теорему доведено.
\end{proof}

\section{Борельові множини}

Клас борельових множин є одним з найважливіших прикладів породжених $\sigma$-алгебр. Нехай на множині $X$ задано метрику $\rho$. Розглянемо метричний простір $(X,\rho)$ і позначимо через $\mathscr G_X$ клас усіх відкритих підмножин $X$.

\begin{definition}
	\label{def-1-8}
	Борельовою $\sigma$-алгеброю в $Y$ (позначається $\BB(X)$) називається $\sigma a\big(\mathscr G_X\big)$. Множини з $\BB(X)$ називаються борельовими.
\end{definition}

\begin{example}
	\label{ex-b-1}
	Будь-яка відкрита множина $U\subset X$ є борельовою тому, що $U\in\mathscr G_X\subset \sigma a\big(\mathscr G_X\big)=\BB\big(X\big)$.
\end{example}

\begin{example}
	\label{ex-b-2}
	Будь-яка замкнена множина $F$ є болельовою.  Дійсно, $X\setminus F$ є відкритою і $X\setminus F\in\mathscr G_X\subset \BB(X)$. Оскільки $\BB(X)$ є $\sigma$-алгеброю, маємо $F=X\setminus(X\setminus F)$.
\end{example}

\begin{example}
	\label{ex-b-3}
	Одноточкова множина є борельовою, оскільки вона є замкненою. 
\end{example}

\begin{example}
	\label{ex-b-4}
	Усі скінченні, зліченні множини та їх доповнення є борельовими, оскільки їх можна подати увигляді скінченного або зліченного об'єднання одноточкових множин та їх доповнення, а $\sigma$-алгебра є замкненою відносно цих операцій. 
\end{example}

Розглянемо теорему про подання борельової $\sigma$-алгебри підмножин $\R^d$.

\begin{theorem}
	\label{th-1-6}
	Для півкільця підмножин $\R^d$
	$$
	\PP_d=\left\{\bigtimes_{k=1}^d (a_k,b_k]\midd \forall k=\overline{1,d}\ -\infty< a_k\leq b_k< +\infty\right\}
	$$
	маємо
	$$
	\sigma k(\PP_d)=\sigma a(\PP_d)=\BB(\R^d).
	$$
\end{theorem}

\begin{proof}
	Оскільки
	$$
	\R^d=\bigcup_{n=1}^\infty (-n,n]^d,
	$$
	маємо $\R^d\in \sigma k(\PP_d)$, отже, 
	\begin{equation}
	\label{1.8}
	\sigma k(\PP_d)=\sigma a(\PP_d).
	\end{equation}

Нехай
$$
A=\bigtimes_{k=1}^d (a_k,b_k]\in\PP_d, \quad A_n=\bigtimes_{k=1}^d \left(a_k,b_k+\frac1n\right),\quad n\in\N.
$$
Тоді $A_n$ є відкритою множиною тому
$A_n\in\BB(\R^d)$, $n\in\N$, і $\displaystyle A=\bigcap_{n=1}^\infty A_n$. Отже, $A\in\BB(\R^d)$ тому $\PP_d\subset\BB(\R^d)$. Звідси випливає 
\begin{equation}
\label{1.9}
\sigma a(\PP_d)\subset\BB(\R^d).
\end{equation}


Залишилося довести протилежне включення. Нехай $U$ є відкритою множиною в $\R^d$. Тоді
\begin{equation}
\label{1.10}
U=\bigcup_{(p,q)\in\mathcal Q} \bigtimes_{k=1}^d (p_k,q_k], 
\end{equation}
де 
\begin{align*}
\mathcal Q&= \left\{(u,v)=\left(\begin{pmatrix}u_1\\ \vdots\\ u_d \end{pmatrix}, \begin{pmatrix}v_1\\ \vdots\\ v_d \end{pmatrix}\right)\in\overline{\Q}{}^d\times \overline{\Q}{}^d \midd 
\right.
\\
&\kern20ex\left. \vphantom{\begin{pmatrix}u_1\\ \vdots\\ u_d \end{pmatrix}}
\bigtimes\limits_{k=1}^d (u_k,v_k]\subset U \land \big(\forall k=\overline{1,d}\   u_k<v_k\big)\right\}.
\end{align*} 
Зрозуміло, що права частина \eqref{1.10} є об'єднанням підмножин $U$, тому сама є підмножиною $U$. З іншого боку, кожна точка відкритої множини $U$ має окіл вигляду  $\bigtimes\limits_{k=1}^d (p_k,q_k]$, $p_k\in\Q$, $q_k\in\Q$, $k=\overline{1,d}$, який міститься в $U$. Тому $U$ міститься об'єднанні, що стоїть у правій частині \eqref{1.10}. Таким чином рівність \eqref{1.10} має місце.

Об'єднання в правій частині \eqref{1.10} є зліченним, тому $U\in \sigma a (\PP_d)$. Отже,  $\mathscr G_\R\subset \sigma a (\PP_d)$, тому 
\begin{equation}
\label{1.11}
\BB\big(\R^d\big)=\sigma a \big(\mathscr G_\R\big)\subset \sigma a \big(\PP_d\big).
\end{equation}
Зі співвідношень  \eqref{1.8}, \eqref{1.9} і \eqref{1.11} випливає твердження теореми.	
\end{proof}


%
%\chapter{ПРОТОКОЛ СИМЕТРИЧНОЇ КРИПТОСИСТЕМИ}\label{ch:-3:---}

Спираючись на теоретичні основи скінченних кілець, відображень та систем лінійних рівнянь, розглянуті у Розділі~2, у цьому розділі подано конкретний протокол обміну інформацією для симетричної криптосистеми.
Спочатку окреслюється необхідний етап налаштування, включаючи початкову домовленість щодо секретних параметрів і генерацію потрібних алгебраїчних компонентів.
Далі наведено покроковий опис дій, які виконують сторони, що обмінюються повідомленнями (традиційно — Аліса і Боб), під час шифрування та дешифрування.
Ілюстративний приклад демонструє роботу протоколу на конкретних параметрах.
Після цього розглядається криптоаналіз протоколу, аналізуються можливі вектори атак і оцінюється стійкість на основі аргументів, наведених у попередніх розділах.
Нарешті, обговорюються практичні аспекти реалізації, обчислювальна ефективність і можливі варіації чи вдосконалення основного протоколу.


\section{Огляд протоколу та етап налаштування}
\label{sec:protocol_overview}
Основна мета протоколу — забезпечити захищений обмін повідомленнями між двома сторонами, Алісою і Бобом, які мають спільний набір початкових секретних параметрів.
Стійкість ґрунтується на симетричній природі ключового матеріалу, отриманого з цих параметрів, і обчислювальній складності для зловмисника відновити відкритий текст із перехопленого шифротексту без знання цих секретів.
Загальна архітектура, що включає декілька кілець ($Z_m, G_m, G_k$) та відображень ($\varphi, \psi, \lambda, \psi_1, \lambda_1$), як показано на рис.~\ref{fig:schema} і розглянуто у~\ref{subsec:system_diagram}, забезпечує основу для роботи протоколу, дозволяючи виконувати обчислення в ефективних доменах, а зовнішньо представляти дані в обфускованому вигляді.

\subsection{Початкова домовленість щодо ключа та генерація параметрів}
\label{subsec:protocol_setup}

Перш ніж розпочати захищене спілкування, Аліса і Боб повинні встановити спільний секретний ключ.
Це досягається на етапі налаштування, що вимагає попереднього захищеного каналу (наприклад, фізичний обмін, захищений протокол узгодження ключа~\cite{DiffieHellman76}).
Під час цього етапу вони погоджують такі початкові секретні параметри:
\begin{itemize}
    \item Параметри $(a, c, l, k)$, необхідні для алгоритму GEN-G (див.~\ref{subsec:gen_g_algorithm}).
    Тут $k$ — порядок більшого кільця $G_k$, $m = k/l$ — порядок робочого кільця $G_m$, а $a, c$ — коефіцієнти, для яких $\gcd(a, k) = 1$.
    \item Конкретні, погоджені правила перетворення, які застосовуються на відповідному кроці алгоритму GEN-G.
    Це критично для отримання ідентичних структур обома сторонами.
\end{itemize}
Використовуючи ці спільні секрети, Аліса і Боб незалежно виконують алгоритм GEN-G:
\begin{itemize}
    \item Вони запускають GEN-G$(a, c, l, m)$ (або адаптують запуск для $k$) з погодженими правилами перетворення для генерації визначального рядка для $G_m$.
    Це неявно визначає кільце $G_m$ та ключовий ізоморфізм $\varphi: Z_m \to G_m$.
    \item Вони запускають GEN-G$(a, c, l, k)$ з погодженими правилами для генерації визначального рядка для більшого кільця $G_k$.
\end{itemize}
Оскільки використовуються однакові вхідні дані та детермінований алгоритм, включаючи ідентичні секретні правила перетворення, обидві сторони гарантовано отримують однакові кільця $G_m, G_k$ та один і той самий ізоморфізм $\varphi$.

Окрім кілець та $\varphi$, для повної функціональності протоколу необхідно також визначити інші відображення, показані на рис.~\ref{fig:schema}: сюр'єкції $\psi, \lambda: G_k \to G_m$ та відповідні бієкції фактор-множин $\psi_1: G_k/\psi \to G_m$, $\lambda_1: G_k/\lambda \to G_m$.
Ці відображення також мають бути погоджені між Алісою і Бобом на етапі налаштування або детерміновано виведені зі спільних секретів $(a,c,l,k)$ за попередньо узгодженим правилом.
Крім того, секретні параметри для афінного перетворення, яке використовує Аліса (матриці $B_i$ та вектори $a_j$), також мають бути встановлені як частина спільного ключового матеріалу.
Повний спільний секретний ключ таким чином складається з $(a,c,l,k)$, правил GEN-G, визначень $\psi, \lambda, \psi_1, \lambda_1$ та послідовності $(B_i, a_j)$. Огляд етапу налаштування та генерації спільного ключа представлено на рис.~\ref{fig:setup_overview}.

\begin{figure}[ht]
    \centering
    \includegraphics[width=0.3\textheight,keepaspectratio]{pictures/Setup Phase Overview Diagram}
    \caption{Огляд етапу налаштування та генерації спільного ключа.}
    \label{fig:setup_overview}
\end{figure}

\subsection{Вибір шляху комунікації}
\label{subsec:communication_paths}

Архітектура системи, зображена на рис.~\ref{fig:schema}, дозволяє гнучко використовувати відображення для зовнішнього представлення.
Можливі три основні шляхи створення шифрограми залежно від того, які фактор-множини використовуються для представлення публічних систем і фінального шифротексту:
\begin{enumerate}
    \item \textbf{Шлях 1:} $G_k/\psi \to G_m \to Z_m \to G_m$.
    Публічні системи $l(x), L(x)$ представлені у $G_k/\psi$.
    Боб переводить їх через $\psi_1^{-1}$ та $\varphi^{-1}$ у $Z_m$ для обчислень.
    Отримані компоненти шифротексту $d, d_1$ (у $Z_m$) відображаються через $\varphi$ у $G_m$ для передачі (або далі).
    \item \textbf{Шлях 2:} $G_k/\lambda \to G_m \to Z_m \to G_k/\psi$.
    Публічні системи $l(x), L(x)$ представлені у $G_k/\lambda$.
    Боб переводить їх через $\lambda_1^{-1}$ та $\varphi^{-1}$ у $Z_m$ для обчислень.
    Отримані компоненти шифротексту $d, d_1$ (у $Z_m$) відображаються через $\varphi$ у $G_m$, далі через $\psi_1$ у $G_k/\psi$ для передачі як представників із $G_k$.
    \item \textbf{Шлях 3:} $G_k/\psi \to G_m \to Z_m \to G_k/\lambda$.
    Публічні системи $l(x), L(x)$ представлені у $G_k/\psi$.
    Боб переводить їх через $\psi_1^{-1}$ та $\varphi^{-1}$ у $Z_m$ для обчислень.
    Отримані компоненти шифротексту $d, d_1$ (у $Z_m$) відображаються через $\varphi$ у $G_m$, далі через $\lambda_1$ у $G_k/\lambda$ для передачі як представників із $G_k$.
\end{enumerate}
Вибір шляху впливає на те, які саме відображення ($\lambda_1^{-1}$ чи $\psi_1^{-1}$ для вхідних даних, $\psi_1$ чи $\lambda_1$ для вихідних) використовуються на інтерфейсах.
Детальний опис протоколу далі відповідає Шляху~2, згідно з основним викладом у попередніх розділах.


\section{Деталізовані кроки протоколу}
\label{sec:protocol_steps}
Після етапу налаштування протокол передбачає три основні інтерактивні кроки.
Нижче описано дії Аліси (яка готує системи та дешифрує) і Боба (який шифрує повідомлення).

\subsection{Крок 1: Аліса — побудова систем і публікація}
\label{subsec:protocol_step1_alice}
Аліса готує основу для обміну, налаштовуючи та публікуючи необхідні системи, використовуючи спільний секретний ключовий матеріал.
\begin{enumerate}
    \item[\textbf{1a.}] \textbf{Побудова базової системи $l(x)$:} Аліса визначає початкову лінійну систему $l(x) = Ax$ над кільцем $G_m$.
    Тут $A$ — матриця розмірності $p \times q$, де $p$ — розмір блоку відкритого тексту, а $q \geq p$ — кількість змінних.
    Аліса повинна забезпечити, щоб $A$ задовольняла умови розв'язності, описані у~\ref{subsec:sle_solvability_rings}: рядки $A$ мають бути лінійно незалежними за модулем $m$, а в $A$ має бути обернена підматриця $A_1$ розмірності $p \times p$.
    Перевірка незалежності здійснюється розв'язанням $A^T y \equiv 0 \pmod{m}$ (єдиний розв'язок — нульовий).
    Далі визначаються $p$ лінійно незалежних стовпців, що утворюють підматрицю $A_1$ з $\gcd(\det(A_1), m) = 1$.
    \item[\textbf{1b.}] \textbf{Застосування секретних афінних перетворень:} Аліса застосовує послідовність з $r \geq 1$ секретних афінних перетворень до $l(x)$, отримуючи перетворену систему $L(x)$.
    Для цього вона використовує секретну послідовність обернених матриць $B_1, \ldots, B_r$ розмірності $p \times p$ та секретні вектори $a_1, \ldots, a_{r+1}$ розмірності $p \times 1$ (усі над $G_m$):
    \[
        L(x) = B_r(B_{r-1}(\ldots B_2(B_1(l(x)+a_1)+a_2)\ldots + a_{r-1})+a_r)+a_{r+1}
    \]
    Обчислення виконуються у $G_m$, зазвичай через ізоморфізм $\varphi$ у $Z_m$.
    В результаті $L(x)$ має афінний вигляд $L(x) = Bx + a$ для деяких $B, a$.
    \item[\textbf{1c.}] \textbf{Обфускація систем для публікації:} Аліса готує системи до публікації.
    Відповідно до обраного шляху (наприклад, Шлях~2), вона використовує відображення $\lambda_1: G_k/\lambda \to G_m$.
    Для кожного коефіцієнта у матрицях $A, B$ та кожної компоненти вектора $a$ (усі в $G_m$) вона знаходить відповідний клас у $G_k/\lambda$ через $\lambda_1^{-1}$ і вибирає представника з цього класу у $G_k$.
    Системи, утворені цими представниками, позначаються $\bar{l}(x) = \bar{A}x$ та $\bar{L}(x) = \bar{B}x + \bar{b}$.
    \item[\textbf{1d.}] \textbf{Публікація систем:} Аліса робить обфусковані системи $\bar{l}(x)$ і $\bar{L}(x)$ загальнодоступними, наприклад, надсилає їх Бобу відкритим каналом або розміщує на вебсайті.
\end{enumerate}

\subsection{Крок 2: Боб — шифрування та передача}
\label{subsec:protocol_step2_bob}
Боб використовує публічні системи Аліси та спільні секрети для шифрування свого повідомлення.
\begin{enumerate}
    \item[\textbf{2a.}] \textbf{Відновлення систем у $Z_m$:} Боб отримує публічні системи $\bar{l}(x) = \bar{A}x$ і $\bar{L}(x) = \bar{B}x + \bar{b}$ з коефіцієнтами у $G_k$.
    Використовуючи спільні секретні відображення для обраного шляху, він спочатку застосовує сюр'єкцію $\lambda: G_k \to G_m$ (відповідає $\lambda_1^{-1}$), а потім ізоморфізм $\varphi^{-1}: G_m \to Z_m$, отримуючи системи у $Z_m$: $\hat{l}(x) = \hat{A}x$, $\hat{L}(x) = \hat{B}x + \hat{a}$.
    \item[\textbf{2b.}] \textbf{Шифрування блоку повідомлення:} Боб перетворює своє повідомлення у послідовність числових значень (наприклад, за табл.~\ref{tab:alphabet}) і ділить на блоки $v$ довжини $p$.
    Для кожного блоку $v$ він виконує:
    \begin{itemize}
        \item Розв'язує базову систему $\hat{l}(x) = v$ ($\hat{A}x \equiv v \pmod{m}$), знаходячи розв'язок $\bar{x} \in Z_m^q$.
        \item Вибирає новий випадковий вектор $\bar{a} \in Z_m^q$, незалежно для кожного блоку.
        \item Обчислює першу компоненту шифротексту $d = \hat{l}(\bar{a}) = \hat{A}\bar{a} \pmod{m}$.
        \item Обчислює другу компоненту шифротексту $d_1 = \hat{L}(\bar{x} + \bar{a}) = \hat{B}(\bar{x} + \bar{a}) + \hat{a} \pmod{m}$.
    \end{itemize}
    \item[\textbf{2c.}] \textbf{Обфускація та передача шифротексту:} Боб зберігає блок відкритого тексту $v$ у таємниці.
    Він обфускує обчислені компоненти шифротексту $d, d_1$ (вектори у $Z_m$) для передачі.
    Відповідно до обраного шляху, спочатку відображає їх у $G_m$ через $\varphi$, далі через бієкцію $\psi_1: G_k/\psi \to G_m$ знаходить відповідні класи у $G_k/\psi$, вибирає представників з цих класів у $G_k$ для формування переданих векторів $(\bar{d}, \bar{d}_1)$.
    Цю пару Боб надсилає Алісі відкритим каналом.
    Кроки 2b і 2c повторюються для кожного блоку повідомлення, причому для кожного блоку використовується новий випадковий вектор $\bar{a}$.
\end{enumerate}

\subsection{Крок 3: Аліса — дешифрування}
\label{subsec:protocol_step3_alice}
Аліса отримує послідовність пар шифротексту від Боба і використовує свій секретний ключовий матеріал для дешифрування кожної з них.
\begin{enumerate}
    \item[\textbf{3a.}] \textbf{Відновлення основних компонент шифротексту:} Для кожної отриманої пари $(\bar{d}, \bar{d}_1)$ (елементи $G_k$) Аліса використовує зворотні відображення для обраного шляху.
    Вона застосовує сюр'єкцію $\psi: G_k \to G_m$ (концептуально $\psi_1^{-1}$), а потім ізоморфізм $\varphi^{-1}: G_m \to Z_m$, отримуючи вектори $d, d_1$ у $Z_m$.
    \item[\textbf{3b.}] \textbf{Обчислення обернених перетворень:} Аліса знає секретні матриці $B_1, \ldots, B_r$, які використовувалися в афінному перетворенні.
    Вона обчислює їх обернені $B_1^{-1}, \ldots, B_r^{-1}$ у $G_m$ або, частіше, обернені $\hat{B}_i^{-1}$ їхніх еквівалентів у $Z_m$ ($\hat{B}_i = \varphi^{-1}(B_i)$) за допомогою ефективних алгоритмів (див.~\ref{subsec:matrix_ops_zm}).
    \item[\textbf{3c.}] \textbf{Зворотне перетворення та відновлення відкритого тексту:} Аліса виконує дешифрувальні обчислення, описані у~\ref{subsec:decryption_mechanism}.
    Використовуючи обернені матриці $\hat{B}_i^{-1}$ та відомі секретні вектори $\hat{a}_j = \varphi^{-1}(a_j)$, вона зворотно відтворює афінне перетворення, застосоване до $d_1$, отримуючи проміжне значення $v_{intermediate} = \hat{l}(\bar{x} + \bar{a}) + \hat{a}_1$.
    Далі вона обчислює остаточний результат:
    \[
        v = v_{intermediate} - (d + \hat{a}_1)
    \]
    Як показано раніше, це дає $v = \hat{l}(\bar{x})$, тобто початковий блок відкритого тексту.
    Аліса повторює цей процес для всіх отриманих пар шифротексту, відновлюючи повне повідомлення.
\end{enumerate}
Коректність цього процесу дешифрування випливає з властивостей афінних перетворень та лінійності $l(x)$, як було доведено у~\ref{sec:core_mechanism}.


\section{Ілюстративний приклад}
\label{sec:protocol_example}
Для наочності роботи протоколу розглянемо детальний числовий приклад, що демонструє всі основні етапи налаштування, шифрування та дешифрування.

\textbf{Етап налаштування:}
\begin{itemize}
    \item \textbf{Початкові параметри:} Аліса і Боб погоджують параметри для GEN-G: $a=7$, $c=5$, $l=2$, $k=50$.
    Модуль робочого кільця $m = k/l = 25$.
    Також погоджуються секретні правила перетворення для кроку~2 GEN-G.
    \item \textbf{Кодування символів:} Для перетворення тексту у числові блоки використовується кодування символів алфавіту згідно з табл.~\ref{tab:alphabet}.
    \item \textbf{Генерація $G_{25}$ та ізоморфізму $\varphi$:} GEN-G з параметрами $(7, 5, 1, 25)$ (або адаптація запуску для $k=50$) і спільними правилами дає визначальний рядок для $G_{25}$:
    \begin{equation*}
      \begin{multlined}
        b_{G_{25}} = (1, 6, 8, 10, 2, 4, 3, 5, 7, 9, 11, 13, 15, 17, 19, \\
                   21, 12, 14, 16, 18, 20, 24, 22, 23, 0)
      \end{multlined}
    \end{equation*}
    Це визначає ізоморфізм $\varphi: Z_{25} \to G_{25}$, наприклад, $\varphi(0)=0$, $\varphi(1)=1$, $\varphi(2)=6$, \ldots, $\varphi(24)=23$.
    \item \textbf{Генерація $G_{50}$:} GEN-G з $(7, 5, 2, 50)$ і спільними правилами дає визначальний рядок для $G_{50}$.
    \item \textbf{Визначення відображень $\psi, \psi_1$:} Погоджується сюр'єкція $\psi: G_{50} \to G_{25}$ та відповідна бієкція $\psi_1: G_{50}/\psi \to G_{25}$.
    Для кожного $j \in \{0, \ldots, 24\}$ (елемент у $G_{25}$ через $\varphi$) вказується два елементи з $G_{50}$ (представники класу у $G_{50}/\psi$).
    \item \textbf{Кодування символів:} Погоджується відповідність символів алфавіту елементам $Z_{25}$ (наприклад, $a=0$, $b=1$, \ldots, $z=24$).
\end{itemize}

\begin{table}[ht]
  \centering
  \small 
  \setlength{\tabcolsep}{3pt}
  \begin{tabular}{|*{25}{c|}}
  \hline 
  0 & 1 & 2 & 3 & 4 & 5 & 6 & 7 & 8 & 9 & 10 & 11 & 12 & 13 & 14  &
  15 & 16 & 17 & 18 & 19 & 20 & 21 & 22 & 23 &  24\\
  \hline
  a & b & c & d & e & f & g & h & i/j & k & l & m & n & o & p & q &
  r & s & t & u & v & w & x & y & z\\
  \hline
  \end{tabular}
  \caption{Цифрові відповідники символів алфавіту}
  \label{tab:alphabet}
\end{table}

Аліса також секретно обирає параметри афінного перетворення: $r=1$, матриця $B_1 = \begin{pmatrix}
                                                                                       6 & 1 \\ 23 & 23
\end{pmatrix}$ у $G_{25}$, вектор $a_1 = (1, 2)^t$ у $G_{25}$.
Вона обчислює еквіваленти у $Z_{25}$: $\bar{B}_1 = \begin{pmatrix}
                                                                                                                      2 & 1 \\ -1 & -1
\end{pmatrix}$, $\hat{a}_1 = (1, 2)^t$, і перевіряє, що $\bar{B}_1$ обернена ($\det(\bar{B}_1) = -1 \equiv 24$, взаємно просте з 25).

\textbf{Виконання протоколу:}

\textbf{Крок 1 (Аліса):}
\begin{itemize}
    \item[\textbf{1a.}] Аліса визначає $l(x) = Ax$ у $G_{25}$. У $Z_{25}$ це $\hat{l}(x) = \hat{A}x$, де
    \[
        \hat{A} = \begin{pmatrix}
                      5 & 6 & 9 & 21 \\ 0 & 1 & 11 & 14
        \end{pmatrix}
    \]
    Вона перевіряє, що матриця задовольняє умови розв'язності (наприклад, стовпці 2 і 3 утворюють підматрицю з визначником $6 \cdot 11 - 9 \cdot 1 = 66 - 9 = 57 \equiv 7 \pmod{25}$, $\gcd(7,25)=1$).
    \item[\textbf{1b.}] Аліса обчислює перетворену систему $L(x) = B_1(l(x) + a_1)$ у $G_{25}$. У $Z_{25}$ це $\hat{L}(x) = \bar{B}_1(\hat{l}(x) + \hat{a}_1)$:
    \[
        \hat{L}(x) = \begin{pmatrix}
                         10x_1 + 13x_2 + 4x_3 + 6x_4 + 7 \\ 20x_1 + 18x_2 + 5x_3 + 15x_4 + 19
        \end{pmatrix}
    \]
    \item[\textbf{1c.}] Аліса обфускує коефіцієнти $l(x)$ і $L(x)$ через відображення у $G_{50}/\psi$, вибираючи представників з $G_{50}$ для кожного коефіцієнта. Наприклад:
    \begin{gather*}
        \bar{l}(x) = \begin{cases}
                         17x_1 + 34x_2 + 21x_3 + 26x_4 \\ 7x_1 + 14x_2 + 42x_3 + 43x_4
        \end{cases}\\
        \bar{L}(x) = \begin{cases}
                         48x_1 + 41x_2 + 3x_3 + 46x_4 + 19 \\ 15x_1 + 32x_2 + 44x_3 + 36x_4 + 30
        \end{cases}\\
    \end{gather*}
    (Коефіцієнти — елементи $G_{50}$, подані як цілі числа 0–49).
    \item[\textbf{1d.}] Аліса публікує $\bar{l}(x)$ і $\bar{L}(x)$.
\end{itemize}

\textbf{Крок 2 (Боб):}
\begin{itemize}
    \item[\textbf{2a.}] Боб отримує $\bar{l}(x), \bar{L}(x)$ і за допомогою зворотних відображень ($\psi^{-1}$, $\varphi^{-1}$) відновлює системи у $Z_{25}$:
    \begin{gather*}
        \hat{l}(x) = \begin{cases}
                         5x_1 + 6x_2 + 9x_3 + 21x_4 \\ 0x_1 + 1x_2 + 11x_3 + 14x_4
        \end{cases}\\
        \hat{L}(x) = \begin{cases}
                         10x_1 + 13x_2 + 4x_3 + 6x_4 + 7 \\ 20x_1 + 18x_2 + 5x_3 + 15x_4 + 19
        \end{cases}\\
    \end{gather*}
    \item[\textbf{2b.}] Боб шифрує повідомлення "tara tara tarara".
    Блоки відкритого тексту (за табл.~\ref{tab:alphabet}): $v_1=(18,0)$ [ta], $v_2=(16,0)$ [ra], $v_3=(18,0)$ [ta], $v_4=(16,0)$ [ra] тощо.

    \textbf{Шифрування блоку 1 ($v_1=(18,0)$):}
    \begin{itemize}
        \item Розв'язує $\hat{l}(x) = (18, 0)$: $\bar{x} = (0, 14, 1, 0)$.
        \item Вибирає випадковий $\bar{a} = (0, 1, 0, 1)$.
        \item Обчислює $d = \hat{l}(\bar{a}) = (5 \cdot 0 + 6 \cdot 1 + 9 \cdot 0 + 21 \cdot 1, 0 \cdot 0 + 1 \cdot 1 + 11 \cdot 0 + 14 \cdot 1) = (27, 15) \equiv (2, 15) \pmod{25}$.
        \item $\bar{x} + \bar{a} = (0, 15, 1, 1)$.
        \item $d_1 = \hat{L}(0, 15, 1, 1) = (10 \cdot 0 + 13 \cdot 15 + 4 \cdot 1 + 6 \cdot 1 + 7, 20 \cdot 0 + 18 \cdot 15 + 5 \cdot 1 + 15 \cdot 1 + 19) = (212, 309) \equiv (12, 9) \pmod{25}$.
    \end{itemize}
    \item[\textbf{2c.}] Боб обфускує $d=(2,15)$, $d_1=(12,9)$ (з $Z_{25}$) у $G_{25}$ через $\varphi$, далі у $G_{50}/\psi$ через $\psi_1$, вибирає представників $\bar{d}, \bar{d}_1$ з $G_{50}$ і передає їх Алісі.
\end{itemize}

\textbf{Крок 3 (Аліса — дешифрування блоку 1):}
\begin{itemize}
    \item[\textbf{3a.}] Аліса отримує обфусковану пару, застосовує зворотні відображення ($\psi^{-1}$, $\varphi^{-1}$), відновлює $d = (2, 15)$, $d_1 = (12, 9)$ у $Z_{25}$.
    \item[\textbf{3b.}] Обчислює обернену до $\bar{B}_1$ у $Z_{25}$:
    \[
        \bar{B}_1^{-1} = \begin{pmatrix}
                             1 & 1 \\ -1 & -2
        \end{pmatrix} \pmod{25}
    \]
    \item[\textbf{3c.}] Виконує дешифрування з використанням $\hat{a} = (7, 19)^t$:
    \begin{itemize}
        \item $d_1' = d_1 - \hat{a} = (12, 9) - (7, 19) = (5, -10) \equiv (5, 15) \pmod{25}$.
        \item $v_{intermediate} = \bar{B}_1^{-1} d_1' = (1 \cdot 5 + 1 \cdot 15, -1 \cdot 5 - 2 \cdot 15) = (20, -35) \equiv (20, 15) \pmod{25}$.
        \item $v = v_{intermediate} - d = (20, 15) - (2, 15) = (18, 0) \pmod{25}$.
    \end{itemize}
    Аліса відновлює $v_1 = (18, 0)$, що відповідає "ta".
    На рисунку~\ref{fig:data_trace_example} показано приклад трасування перетворень для одного числового значення через різні кільця та відображення.
\end{itemize}

\begin{figure}[ht]
    \centering
    \includegraphics[width=\textwidth]{pictures/Example Data Trace Diagram}
    \caption{Приклад трасування перетворень для одного числового значення.}
    \label{fig:data_trace_example}
\end{figure}

\textbf{Шифрування/дешифрування наступних блоків:}
\begin{itemize}
    \item \textbf{Блок 2 ("ra", $v_2=(16,0)$):}
    \begin{itemize}
        \item $\bar{x}=(0,18,12,0)$, $\bar{a}=(1,0,1,0)$.
        \item $d=(14,11)$, $d_1=(3,3)$.
        \item $\bar{B}_1^{-1}(d_1-\hat{a})^t = (5,11)$, $v = (5,11)-(14,11) = (-9,0) \equiv (16,0)$.
    \end{itemize}
    \item \textbf{Блок 3 ("ta", $v_3=(18,0)$), новий $\bar{a}$:}
    \begin{itemize}
        \item $\bar{x}=(0,14,1,0)$, $\bar{a}=(0,0,1,1)$.
        \item $d=(5,0)$, $d_1=(3,21)$.
        \item $\bar{B}_1^{-1}(d_1-\hat{a})^t = (23,0)$, $v = (23,0)-(5,0) = (18,0)$.
    \end{itemize}
    \item \textbf{Блок 4 ("ra", $v_4=(16,0)$), новий $\bar{a}$:}
    \begin{itemize}
        \item $\bar{x}=(0,18,12,0)$, $\bar{a}=(0,0,0,1)$.
        \item $d=(21,14)$, $d_1=(20,18)$.
        \item $\bar{B}_1^{-1}(d_1-\hat{a})^t = (12,14)$, $v = (12,14)-(21,14) = (-9,0) \equiv (16,0)$.
    \end{itemize}
\end{itemize}
Процедура повторюється для всіх блоків.
Фінальна послідовність пар шифротексту $(d, d_1)$ у $Z_{25}$: $((2,15), (12,9)), ((14,11), (3,3)), ((5,0), (3,21)), ((21,14), (20,18)), \ldots$.
Після дешифрування всіх пар Аліса відновлює числову послідовність відкритого тексту $(18,0), (16,0), (18,0), (16,0), \ldots$ і перетворює її у повідомлення "tara tara tarara".

Цей приклад підкреслює необхідність використання нового випадкового вектора $\bar{a}$ для кожного блоку, навіть для повторюваних блоків відкритого тексту, щоб забезпечити різні шифротексти.
Також продемонстровано, що дешифрування успішно відновлює відкритий текст шляхом зворотного афінного перетворення із застосуванням секретної оберненої матриці $\bar{B}_1^{-1}$ та віднімання випадкової компоненти $d$.


\section{Криптоаналіз протоколу}
\label{sec:protocol_cryptanalysis}
У цьому підрозділі аналізується стійкість запропонованого протоколу до основних типів криптоаналітичних атак.
Розглядається точка зору зовнішнього атакуючого, який має доступ до публічної інформації (системи, шифротексти) та знає загальний алгоритм, але не володіє секретними параметрами.

\subsection{Інформація, доступна атакуючому}
\label{subsec:attacker_knowledge}
Відповідно до принципу Керкгоффса, атакуючий знає всі алгоритми, але не конкретний секретний ключ.
Згідно з описом протоколу (див.~\ref{sec:protocol_steps}), атакуючому доступна така інформація:
\begin{enumerate}
    \item \textbf{Публічні системи:} Обфусковані системи $\bar{l}(x)$ та $\bar{L}(x)$ з коефіцієнтами у $G_k$ (тобто числа від $0$ до $k-1$). З $\bar{L}(x)$ можна визначити кількість рівнянь $p$ та кількість змінних $q$.
    \item \textbf{Шифротексти:} Послідовність пар $(\bar{d}_i, \bar{d}_{i1})$, де кожна компонента — елемент $G_k$.
    \item \textbf{Порядки кілець:} Можливо, відомі або визначаються порядки $m$ і $k$ кілець $Z_m$ ($G_m$) та $Z_k$ ($G_k$).
    \item \textbf{Структура протоколу:} Відомі ролі $l(x), L(x)$, випадкового вектора $\bar{a}$, відображень тощо.
\end{enumerate}
Недоступною для атакуючого є така інформація:
\begin{enumerate}
    \item Конкретні параметри $(a, c)$ і секретні правила перетворення для GEN-G.
    \item Відповідний ізоморфізм $\varphi: Z_m \to G_m$.
    \item Сюр'єкції $\psi, \lambda: G_k \to G_m$.
    \item Бієкції $\psi_1: G_k/\psi \to G_m$, $\lambda_1: G_k/\lambda \to G_m$.
    \item Секретні матриці $B_i$ і вектори $a_j$ для афінного перетворення.
    \item Випадкові вектори $\bar{a}_i$ для кожного блоку.
    \item Блоки відкритого тексту $v_i$.
\end{enumerate}

\subsection{Можливі вектори атак}
\label{subsec:attack_vectors}
Розглянемо основні типи атак, які може застосувати атакуючий.

\subsubsection{Атака повним перебором}
\label{subsubsec:brute_force}

Атака полягає у систематичному переборі всіх можливих комбінацій секретних компонентів ключа для відновлення відкритого тексту.

Загальна кількість комбінацій для лише відображень оцінюється як
\[
(m-2)!\cdot m! \cdot \frac{k!}{m!(l!)^m},
\]
де $(m-2)!$ — кількість можливих ізоморфізмів $\varphi$, $m!$ — кількість бієкцій $\psi_1$ (або $\lambda_1$), а $\frac{k!}{m!(l!)^m}$ — кількість сюр'єкцій $\psi$ (або $\lambda$), де $k=lm$.

Навіть для відносно невеликих параметрів ця кількість є астрономічною.
Наприклад, для $m=25$, $k=50$, $l=2$:
\begin{equation*}
    23! \cdot 25! \cdot \frac{50!}{25! \cdot 2^{25}} = \frac{23! \cdot 50!}{2^{25}} > 2^{94} > 10^{31}\, \text{сек}.
\end{equation*}
Якщо припустити, що одна комбінація перевіряється за $10^{-14}$ секунд, то повний перебір займе понад $10^{31}$ секунд, тобто більше ніж $10^7$ років.

Додавання перебору афінних параметрів ($B_i, a_j$) ще більше збільшує складність.
Таким чином, атака повним перебором є обчислювально нездійсненною навіть для невеликих порядків кілець; збільшення $m$ і $k$ робить її ще менш реальною.

\subsubsection{Атаки з відомим відкритим текстом або шифротекстом}
\label{subsubsec:known_plaintext}

Розглянемо два сценарії:

\begin{enumerate}[label=(\alph*)]
    \item \textbf{Атака лише за шифротекстом.} \\
    Атакуючий має доступ до багатьох пар шифротексту $(\bar{d}_i, \bar{d}_{i1})$ для невідомих відкритих текстів. Відомо, що ці пари пов'язані з внутрішніми значеннями $(d_i, d_{i1})$ через невідомі відображення ($\varphi, \psi_1$ тощо). Однак для відновлення відкритого тексту необхідно знати ці відображення та секретні перетворення ($B_i, a_j$). Простір можливих відображень має розмір, пов'язаний з $(m-2)! \times m! \times \frac{k!}{m!(l!)^m}$, і навіть наявність багатьох шифротекстів не зменшує суттєво складність пошуку потрібних секретних відображень. Шифротексти у $G_k$ не дають прямої інформації про структуру $G_m$ чи $Z_m$ без правильних зворотних відображень.

    \item \textbf{Атака з відомим відкритим текстом.} \\
    Атакуючий має декілька пар відомих блоків відкритого тексту $v_i$ та відповідних обфускованих шифротекстів $(\bar{d}_i, \bar{d}_{i1})$. Зв'язок між ними задається через невідомі відображення ($\xi, \varphi, \psi_1$), і атакуючий не може легко отримати основні компоненти шифротексту $(d_i, d_{i1})$ у $Z_m$. Навіть якщо це вдасться, рівняння $d_i = \hat{l}(\bar{a}_i)$, $d_{i1} = \hat{L}(\bar{x}_i + \bar{a}_i)$, де $v_i = \hat{l}(\bar{x}_i)$, містять невідомий випадковий вектор $\bar{a}_i$ для кожної пари. Це унеможливлює складання простої системи рівнянь для відновлення коефіцієнтів $\hat{l}$, $\hat{L}$ чи параметрів перетворення. Відповідно, навіть за наявності пар відкритий текст — шифротекст, атакуючий не може відновити визначальний рядок ($\varphi$) чи секретні перетворення через поєднання невідомих відображень і випадковості для кожного шифрування~\cite{KatzLindell14}.
\end{enumerate}

\subsubsection{Частотний аналіз та статистичні атаки}
\label{subsubsec:frequency_analysis}
Як показано у~\ref{subsec:randomness_mappings_role}, протокол стійкий до частотного аналізу. Це забезпечується використанням нового випадкового вектора $\bar{a}$ для кожного блоку: однакові блоки відкритого тексту $v$ шифруються у різні пари $(d, d_1)$ (і, відповідно, різні $(\bar{d}, \bar{d}_1)$) при кожному шифруванні. Така ймовірнісна схема руйнує детермінований зв'язок між відкритим і шифрованим текстом, на якому ґрунтується частотний аналіз. Додатковий рівень обфускації через фактор-множини $G_k/\psi$ чи $G_k/\lambda$ ще більше приховує статистичні властивості основних компонент шифротексту~\cite{StinsonPaterson18}.

\subsubsection{Атаки, що базуються на груповій структурі (циклічність)}
\label{subsubsec:group_structure}
У протоколі враховується структура мультиплікативної групи дільників одиниці $Z_k^*$ (див. теорему~\ref{thm:cyclic_units}): $Z_k^*$ є циклічною тоді і тільки тоді, коли $k=2,4,p^m$ або $2p^m$ для непарного простого $p$. Оскільки у протоколі часто використовуються складені модулі $k$ (і $m$), які не підпадають під умови теореми, група дільників одиниці $Z_k^*$ (і $Z_m^*$) зазвичай не є циклічною. Це означає, що стандартні криптоаналітичні атаки, засновані на складності дискретного логарифма (DLP) у циклічних групах, тут не застосовні. Стійкість системи базується на комбінаторній складності відображень і специфічній алгебраїчній структурі перетворень СЛР, а не на класичних групових задачах типу DLP.


\section{Практичні аспекти реалізації та варіації}
\label{sec:implementation_considerations}
Окрім теоретичного опису та аналізу стійкості, практичне впровадження запропонованої криптосистеми вимагає врахування обчислювальної ефективності, можливих удосконалень і варіацій основного протоколу.
У цьому підрозділі розглядаються ці аспекти на основі структури протоколу, ілюстративних прикладів і висновків попередніх розділів.

\subsection{Обчислювальна ефективність}
\label{subsec:computational_efficiency}
Продуктивність криптосистеми визначається ефективністю базових операцій на етапах налаштування, шифрування та дешифрування.
Основні обчислювальні витрати для кожної фази (передбачається виконання у $Z_m$ через ізоморфізм $\varphi$) такі:

\begin{itemize}
    \item \textbf{Етап налаштування (узгодження ключа):}
    \begin{itemize}
        \item \textbf{Виконання GEN-G:} Генерація визначальних рядків для $G_m$ і $G_k$ за допомогою GEN-G має складність $O(m \log^2 m)$ та $O(k \log^2 k)$ відповідно, що визначається кількістю модульних множень. Секретний крок перетворення (крок~2) додає складність залежно від обраних перестановок (наприклад, перемішування — $O(k)$).
        \item \textbf{Попереднє обчислення відображень:} Побудова зворотних відображень ($\varphi^{-1}$, $\psi_1^{-1}$, $\lambda_1^{-1}$ тощо, наприклад, у вигляді таблиць) потребує ітерації по згенерованих структурах, зазвичай $O(m)$ або $O(k)$.
        \item \textbf{Вибір перетворень:} Аліса повинна вибрати секретні матриці $B_i$ та вектори $a_j$. Генерація випадкових обернених матриць $B_i$ може вимагати перевірки визначника. Попереднє обчислення ефективного перетворення $B_{eff} = B_r \dots B_1$ та його оберненого $B_{eff}^{-1} = B_1^{-1} \dots B_r^{-1}$ вимагає $O(r \cdot p^3)$ операцій у кільці. Обчислення накопиченого вектора $a_{outer}$ також включає матрично-векторні множення.
    \end{itemize}
    Етап налаштування може бути обчислювально затратним, але виконується лише раз на сесію.

    \item \textbf{Аліса — крок 1 (налаштування/публікація систем):}
    \begin{itemize}
        \item \textbf{Побудова $l(x)$:} Генерація матриці $A$ розмірності $p \times q$, що задовольняє умови розв'язності, включає генерацію випадкових елементів і перевірку незалежності рядків (розв'язання $A^T y \equiv 0$), пошук оберненої підматриці (перевірка визначників).
        \item \textbf{Обчислення $L(x)$:} Якщо $B_{eff}$ та $a_{outer}$ попередньо обчислені, отримання $L(x) = Bx + a$ — це одне множення $p \times p$ на $p \times q$ ($O(p^2 q)$) та додавання векторів.
        \item \textbf{Відображення коефіцієнтів:} Відображення $pq$ коефіцієнтів $A$ та $pq+p$ коефіцієнтів/констант $L(x)$ у представники $G_k$ через $\lambda_1$ та $\varphi$ — $O(pq)$ операцій.
    \end{itemize}

    \item \textbf{Боб — крок 2 (шифрування блоку):}
    \begin{itemize}
        \item \textbf{Відображення коефіцієнтів:} Відновлення $\hat{l}(x)$ і $\hat{L}(x)$ у $Z_m$ — це $O(pq)$ операцій через $\lambda_1^{-1}$ та $\varphi^{-1}$.
        \item \textbf{Розв'язання $\hat{l}(x)=v$:} Знаходження розв'язку $\bar{x}$ для системи $p \times q$. Якщо використовується попередньо обчислений обернений мінор $A_1^{-1}$, це одне множення $p \times p$ на вектор ($O(p^2)$). Якщо використовується загальний метод (Гаусс), складність може бути $O(p^2 q)$ або $O(p^3)$.
        \item \textbf{Обчислення $d, d_1$:} Множення матриці на вектор ($O(pq)$), додавання векторів ($O(q)$), ще одне множення для афінного відображення ($O(pq)$).
        \item \textbf{Відображення шифротексту:} Відображення $2p$ компонентів через $\varphi$ і $\psi_1$ — $O(p)$ операцій.
    \end{itemize}
    Основна складність — розв'язання СЛР та матрично-векторні множення ($O(pq + p^2)$).

    \item \textbf{Аліса — крок 3 (дешифрування блоку):}
    \begin{itemize}
        \item \textbf{Відображення шифротексту:} Відновлення $d, d_1$ у $Z_m$ — $O(p)$ операцій через $\psi_1^{-1}$ та $\varphi^{-1}$.
        \item \textbf{Обернені перетворення:} Якщо $B_{eff}^{-1}$ попередньо обчислений, витрати мінімальні; інакше — $r$ інверсій ($O(r \cdot p^3)$).
        \item \textbf{Зворотне перетворення та відновлення тексту:} Векторне віднімання, одне множення $p \times p$ на вектор ($O(p^2)$), ще одне віднімання.
    \end{itemize}
    Основна складність — матрично-векторне множення для зворотного перетворення ($O(p^2)$), якщо $B_{eff}^{-1}$ вже є.
\end{itemize}

Найбільш ресурсоємні операції — обернення матриць (на етапі налаштування) та розв'язання СЛР при шифруванні.
Виконання цих операцій у $Z_m$ через ізоморфізм $\varphi$ і стандартні поліноміальні алгоритми (розширений алгоритм Евкліда, Гаусс) є критичним для ефективності.
Складність — поліноміальна відносно $p, q, \log m$.

Важливою особливістю є \textbf{розширення шифротексту}: кожен блок відкритого тексту $v$ (вектор довжини $p$) шифрується у пару векторів $(d, d_1)$ (кожен довжини $p$), тобто шифротекст у 2 рази довший за відкритий текст (без урахування можливого збільшення розміру через представлення елементів $Z_m$ як елементів $G_k$).

\subsection{Можливі удосконалення та варіації}
\label{subsec:protocol_variations}
Основний протокол може бути модифікований або розширений для досягнення різних компромісів між стійкістю та продуктивністю.
Можливі такі варіанти:

\begin{itemize}
    \item \textbf{Оновлення параметрів:} Замість постійного використання одних і тих самих кілець ($G_m, G_k$) і параметрів ($B_i, a_j$), їх можна періодично змінювати або для кожної сесії. Це передбачає повторний запуск етапу налаштування (узгодження нових параметрів GEN-G чи правил перетворення) для генерації нового ключа. Такий підхід обмежує обсяг шифротексту під одним ключем і зменшує ризик при компрометації.
    \item \textbf{Випадковість для кожного блоку:} Критично важливо використовувати криптографічно стійкий генератор випадкових чисел для генерації нового, непередбачуваного вектора $\bar{a}$ для \emph{кожного} блоку. Повторне використання або передбачуваність $\bar{a}$ може створити алгебраїчні зв'язки між шифротекстами і відкрити шлях до атаки.
    \item \textbf{Вибір параметрів ($p, q, m, k, r$):} Збільшення $m$ і $k$ підвищує стійкість до перебору, але збільшує обчислювальні витрати. Збільшення $p$ зменшує кількість шифрувань для довгого повідомлення, але підвищує вартість одного блоку (матриці розмірності $p$). Збільшення $q$ дає більше простору для випадковості, але ускладнює обчислення. Збільшення $r$ (кількість афінних перетворень) ускладнює зв'язок між $l(x)$ і $L(x)$, але підвищує витрати на налаштування і дешифрування. Необхідний баланс між стійкістю та продуктивністю.
\end{itemize}
Ці варіації дозволяють адаптувати протокол до конкретних вимог щодо стійкості та продуктивності, але кожна модифікація потребує окремого аналізу безпеки.

%
%\chapter*{Висновки}
\markboth{Висновки}{Висновки}
\addcontentsline{toc}{chapter}{Висновки}

\hspace{\parindent}Тут ви можете навести висновки до своєї роботи.
%
%\begin{thebibliography}{XX}
	\addcontentsline{toc}{chapter}{Список використаних джерел}

\bibitem{AD1}
Дороговцев А.Я. Математичний аналіз. Ч. 1. Київ : Либідь, 1993.
	
\bibitem{AD2}
Дороговцев А.Я. Математичний аналіз. Ч. 2. Київ : Либідь, 1994.

\bibitem{VR}
Радченко В.М. Теорія міри та інтеграла. Київ : Видавничо-полі\-гра\-фіч\-ний  центр ``Київський університет'', 2012.

\bibitem{VIK}
Korobov V.I. Time optimality for systems with multidimensional control and vector moment min-problem. \emph{Journal of Dynamical and Control Systems}. 2020. V.~26(3). P.~525--550.

\end{thebibliography}


\end{document}




